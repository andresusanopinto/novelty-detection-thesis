\documentclass[runningheads,a4paper]{llncs}

\usepackage{amssymb}
\setcounter{tocdepth}{3}
\usepackage{graphicx}

\usepackage{url}
\newcommand{\keywords}[1]{\par\addvspace\baselineskip
\noindent\keywordname\enspace\ignorespaces#1}

% Additional packages
\usepackage[utf8]{inputenc}

\begin{document}

\mainmatter  % start of an individual contribution

% first the title is needed
\title{Using Graphical Models for Novelty Detection on Semantic Room Classification}

% a short form should be given in case it is too long for the running head
% \titlerunning{Lecture Notes in Computer Science: Authors' Instructions}

% the name(s) of the author(s) follow(s) next
%
% NB: Chinese authors should write their first names(s) in front of
% their surnames. This ensures that the names appear correctly in
% the running heads and the author index.
%
\author{Ommited for Review}
%
%\authorrunning{Lecture Notes in Computer Science: Authors' Instructions}
% (feature abused for this document to repeat the title also on left hand pages)

% the affiliations are given next; don't give your e-mail address
% unless you accept that it will be published
\institute{Institution ommited for Review}

\toctitle{Lecture Notes in Computer Science}
\tocauthor{Authors' Instructions}
\maketitle


\begin{abstract}
This papers presents an approach on implementing novelty detection for indoor room classification
using semantic data.
\emph{Graphical models} are used to model probabilistic knowledge and a novelty threshold is
defined in terms of conditional and unconditional probabilities.
The novelty threshold is then optimized using an unconditional probability density
model trained from unlabelled data.


\keywords{novelty detection, semantic data, probabilistic graphical models,
room classification, indoor environments, robotics, multi-modal classification.}
\end{abstract}

\end{document}
