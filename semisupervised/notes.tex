\documentclass[a4paper]{article}

\usepackage[utf8]{inputenc}
\usepackage[english]{babel}
\usepackage[round]{natbib}
% Additional packages
\usepackage[utf8]{inputenc}
\usepackage{subfig}
\usepackage[pdftex,pdfpagelabels,bookmarks,hyperindex,hyperfigures]{hyperref}
\hypersetup{%
   plainpages=false, 
   pdfpagelayout=SinglePage,
   bookmarksopen=false,
   bookmarksnumbered=true,
   breaklinks=true,
   linktocpage,
   colorlinks=true,
   linkcolor=blue,
   urlcolor=blue,
   citecolor=blue,
   anchorcolor=green
}      

% There seems to exist some kind of standard on using CHapter with capital.
\def\chapterautorefname{Chapter}

\newtheorem{algorithm}{Algorithm}

% Based on: http://lists.cs.princeton.edu/pipermail/topic-models/2010-December/001081.html
% http://www.mpi-inf.mpg.de/~dietz/probabilistic-models-tikz.zip
% by Laura Dietz (dietz at mpi-inf.mpg.de)
%
\usepackage{color}
\usepackage{array}
\usepackage{verbatim}
\usepackage{float}
\usepackage{amsmath}
\usepackage{amssymb}
\usepackage{esint}

%%%%%%%%%%%%%%%%%%%%%%%%%%%%%% LyX specific LaTeX commands.
%% Because html converters don't know tabularnewline
\providecommand{\tabularnewline}{\\}
\floatstyle{ruled}
\newfloat{algorithm}{tbp}{loa}[chapter]
\floatname{algorithm}{Algorithm}

%%%%%%%%%%%%%%%%%%%%%%%%%%%%%% Textclass specific LaTeX commands.
\usepackage{float}
\floatstyle{ruled}
\newfloat{algorithm}{tbp}{loa}
\floatname{algorithm}{Algorithm}
\newfloat{genmodel}{h}{loa}
\floatname{genmodel}{Generative Process}
\usepackage[noend]{algorithmic}
\newcommand{\forbody}[1]{ #1 \ENDFOR}
\newcommand{\ifbody}[1]{ #1  \ENDIF}
\newcommand{\whilebody}[1]{ #1  \ENDWHILE}
\renewcommand{\algorithmicprint}{\textbf{draw}}

%%%%%%%%%%%%%%%%%%%%%%%%%%%%%% User specified LaTeX commands.


\usepackage{euscript}

\DeclareSymbolFont{rsfscript}{OMS}{rsfs}{m}{n}
\DeclareSymbolFontAlphabet{\mathrsfs}{rsfscript}


% PDF formatting instructions for A4
\pdfpagewidth=210mm % for pdflatex
\pdfpageheight=296mm % for pdflatex


%%%%%%%% begin tikz %%%%%%
\usepackage{tikz,tkz-base}
\usetikzlibrary{shapes,decorations,shadows}
\usetikzlibrary{decorations.pathmorphing}
\usetikzlibrary{decorations.shapes}
\usetikzlibrary{fadings}
\usetikzlibrary{patterns}
\usetikzlibrary{calc}
\usetikzlibrary{decorations.text}
\usetikzlibrary{decorations.footprints}
\usetikzlibrary{decorations.fractals}
\usetikzlibrary{shapes.gates.logic.IEC}
\usetikzlibrary{shapes.gates.logic.US}
\usetikzlibrary{fit,chains}
\usetikzlibrary{positioning}
\usepgflibrary{shapes}
\usetikzlibrary{scopes}
\usetikzlibrary{arrows}
\usetikzlibrary{backgrounds}


\pgfdeclarelayer{background}
\pgfdeclarelayer{foreground}
\pgfsetlayers{background,main,foreground}

\tikzset{latent/.style={circle,fill=white,draw=red,thick,inner sep=1pt, 
minimum size=20pt, font=\fontsize{10}{10}\selectfont},
obs/.style={latent,fill=gray!25},
const/.style={rectangle, inner sep=0pt},
factor/.style={rectangle, fill=red,minimum size=7pt, inner sep=0pt},
yellow/.style={latent,minimum size=15pt,fill=yellow!75},
blue/.style={latent,minimum size=15pt,fill=blue!75},
-/.style={color=red, thick},
>={triangle 45}}




% shapename, fitlist, caption, pos
\newcommand{\plate}[4]{
\node (invis#1) [draw, transparent, inner sep=1pt,rectangle,fit=#2] {};
\node (capt#1) [ below left=0 pt of invis#1.south east, xshift=0pt,yshift=-9pt] {\raisebox{0pt}[0pt]{\footnotesize{#3}}};
\node (#1) [draw=black!50,thick,inner sep=3pt,rectangle,rounded corners,fit=(invis#1) (capt#1),#4] {};
}


\newcommand{\shiftedplate}[5]{
\node (invis#1) [draw, transparent, inner sep=0 pt,rectangle,fit=#2] {};
\node (capt#1) [#5, xshift=2pt] {\footnotesize{#3}};
\node (#1) [draw,inner sep=2pt, rectangle,fit=(invis#1) (capt#1),#4] {};
}

%shapename, pos, caption, in1, in2, out, captpos
\newcommand{\twofactor}[7]{
\node (#1) [factor] at #2 {};
\node (capt#1) [#7 of #1]{\footnotesize{#3}};
\draw [-] (#4) -- (#1) ;
\draw [-] (#5) -- (#1) ;
\draw [->,thick] (#1) -- (#6);
}

%shapename, pos, caption, in, out, captpos
\newcommand{\factor}[6]{
\node (#1) [factor] at #2 {};
\node (capt#1) [#6 of #1]{\footnotesize{#3}};
\draw [-] (#4) -- (#1) ;
\draw [->,thick] (#1) -- (#5);
}

% name, --, caption, pos
\newcommand{\nofactor}[4]{
\node (#1) [factor, #2]  {};
\node (capt#1) [#4 of #1]{\footnotesize{#3}};
}

%shapename,  fitlist, caption
\newcommand{\namedgate}[3]{
\node (invisgate#1) [rectangle, draw, transparent,  fit=#2] {};
\node (gatecapt#1) [ above right=0 pt of invisgate#1.north west, xshift=-1pt ] {\footnotesize{#3}};
\node (#1) [rectangle,draw,dashed, inner sep=2pt, fit=(invisgate#1)(gatecapt#1)]{};

}

%shapename,  fitlist, caption
\newcommand{\gate}[3]{
\node (#1) [rectangle,draw,dashed, inner sep=2pt, fit=#2]{};
}

%shapename,  fitlist1, fitlist2, caption1, caption2
\newcommand{\vertgate}[5]{
\node (invisgateleft#1) [rectangle, draw, transparent,  fit=#2] {};
\node (gatecaptleft#1) [ above left=0 pt of invisgateleft#1.north east, xshift=1pt ]{\footnotesize{#3}};
\node (invisgateright#1) [rectangle, draw, transparent,  fit=#4] {};
\node (gatecaptright#1) [ above right=0 pt of invisgateright#1.north west, xshift=-1pt ] {\footnotesize{#5}};
\node (#1) [rectangle,draw,dashed, inner sep=2pt, fit=(invisgateleft#1)(gatecaptleft#1)(invisgateright#1)(gatecaptright#1)]{};
\draw [-, dashed] (#1.north) -- (#1.south);
}


\newcommand{\vertgateSpec}[5]{
\node (invisgateleft#1) [rectangle, draw, transparent,  fit=#2] {};
\node (gatecaptleft#1) [ above left=0 pt of invisgateleft#1.north east, xshift=1pt ]{\footnotesize{#3}};
\node (invisgateright#1) [rectangle, draw, transparent,  fit=#4] {};
\node (gatecaptright#1) [ above right=0 pt of invisgateright#1.north west, xshift=-1pt ] {\footnotesize{#5}};
\node (#1) [rectangle,draw,dashed, inner sep=2pt, fit=(invisgateleft#1)(gatecaptleft#1)(invisgateright#1)(gatecaptright#1)]{};
\draw [-, dashed] (#1.70) -- (#1.290);
}

\newcommand{\horgate}[5]{
\node (invisgateleft#1) [rectangle, draw, transparent,  fit=#2] {};
\node (gatecaptleft#1) [ above right=0 pt of invisgateleft#1.south west, xshift=1pt ]{\footnotesize{#3}};
\node (invisgateright#1) [rectangle, draw, transparent,  fit=#4] {};
\node (gatecaptright#1) [ below right=0 pt of invisgateright#1.north west, xshift=-1pt ] {\footnotesize{#5}};
\node (#1) [rectangle,draw,dashed, inner sep=2pt, fit=(invisgateleft#1)(gatecaptleft#1)(invisgateright#1)(gatecaptright#1)]{};
\draw [-, dashed] (#1.west) -- (#1.east);
}

\newcommand{\horogate}[5]{
\node (invisgateleft#1) [rectangle, draw, transparent,  fit=#2] {};
\node (invisgateright#1) [rectangle, draw, transparent,  fit=#4] {};
\node (#1) [rectangle,draw,dashed, inner sep=2pt, fit=(invisgateleft#1)(invisgateright#1)]{};
\node (gatecaptleft#1) [ above right=0 pt of #1.west, xshift=0pt ]{\footnotesize{#3}};
\node (gatecaptright#1) [ below right=0 pt of #1.west, xshift=0pt ] {\footnotesize{#5}};

\draw [-, dashed] (#1.west) -- (#1.east);
}


\newcommand{\vertogate}[5]{
\node (invisgateleft#1) [rectangle, draw, transparent,  fit=#2] {};
\node (invisgateright#1) [rectangle, draw, transparent,  fit=#4] {};
\node (#1) [rectangle,draw,dashed, inner sep=2pt, fit=(invisgateleft#1)(invisgateright#1)]{};
\node (gatecaptleft#1) [ below left=0 pt of #1.north, xshift=0pt ]{\footnotesize{#3}};
\node (gatecaptright#1) [ below right=0 pt of #1.north, xshift=0pt ] {\footnotesize{#5}};

\draw [-, dashed] (#1.north) -- (#1.south);
}



\title{Notes on Semi-Supervised `Novelty Detection'}
\author{André Susano Pinto}

\begin{document}
\maketitle

\section{Introduction}
The case presented here assumes that a machine learning system besides having access to samples of the known class also has access to unlabelled data representative of the whole world.
This may happen in some machine learning problems where a set of samples from a given class has been grouped and identified and although its possible to gather new samples its infeasible to classify them (eg.: the human classifier got bored)\footnote{Other examples of situations are given on \autoref{sec:detection}.}.

Under that assumption its possible to model the functions $P(x|know)$ and $P(x)$ representing both a conditional and an unconditional density function for a given input.
Using those its possible to define a threshold function for novelty detection as seen on \autoref{sec:threshold}.

In \autoref{sec:threshold-analysis} the described threshold is analyzed in more detail, and in \autoref{sec:experiment} experiments to test the described methods are presented.

\subsection{(Novelty?) Detection}
\note{I believe calling \emph{novelty detection} to the threshold described here is incorrect.
I am not sure which would be the correct term.}

This new assumption of access to unlabelled data makes it dubious whether it can still be classified as a \emph{novelty detection} problem:
\begin{quotation}
`Novelty detection is the identification of new or unknown data or signal that a machine learning system is not aware of during training.' - \cite{markou2003novelty}.
\end{quotation}

But it is also different from a binary classification problem since instead of negative samples there is access to unlabelled data\footnote{Though several detection systems use binary classification techniques where negative samples are obtained by randomly sampling.}.


\section{Concept Detection}
\label{sec:detection}

\note{
This lacks literature background, and writing it on an airplane does not helps.
Gotta find some papers on the subject
}

In several machine learning problems the target is to detect whether a concept is present.
(eg.: face detection, intrusion detection).

In cases where negative samples are \emph{abundant} (face detection, music/speech detection), systems often use a random sample of input space to represent negative samples and then perform detection using a normal 2-class classification.

On systems where negative samples are very rare (intrusion-detection, cancer-detection), system need to use other approaches based on novelty detection\dots

Some cases lie somewhere in between: negative samples are easy to obtain, but they are not that common that we can assume a random sample does not contains positive samples in it. Hence we cannot treat a random sample as a negative.

In those cases we point to the usage of the ratio between conditional and unconditional probabilities as a detection function.
Examples of cases where this type of detection is may or may not appear are given below:

\subsubsection*{Unconstrainted labelled data}
On a system where a user is asked to tag samples with words representing concepts it is fairly easy to filter positive samples for a given concept.
Although getting negative samples is an impossible task, as the non-presence of a given tag on a sample does not necessarily makes that a negative sample for the concept.

Eg.: associating concepts with images, tagging objects on images, ...

\subsubsection*{Tag-prediction systems}
Music, image, text, news, auto-tagging\dots

\subsubsection*{Good approximation of density functions}
A good approximation of probability functions for a given input is necessary for the performance of the presented method.
So its expected to have uses where the input space is small or it can be modelled with some precision (eg.: independent features).

\section{Threshold}
\label{sec:threshold}

The threshold is based on the probability that a given concept is present on a sample
knowing both that the concept samples and the unlabelled samples come from the same distribution $P$.

In that case our optimal threshold would be:
$P(concept | x) > 1/2$, which means we assume the concept is present if we have more than chance probability of getting it right.

Using bayes rules we obtain:

\begin{equation}\frac{P(x|concept) P(concept)}{P(x)} > \frac{1}{2}\end{equation}
\begin{equation}\frac{P(x|concept)}{P(x)} > \frac{1}{2 P(concept)}\end{equation}
\begin{equation}\frac{P(x|concept)}{P(x)} > K_{threshold}\end{equation}

\note{Novelty detection techniques often assume a constant $P(x)$ and then threshold is performed only on the conditional probability density function.}

\section{Threshold Analysis}
\label{sec:threshold-analysis}

\begin{equation}P(x) = \sum_{c}^{} P(x|c) P(c)\end{equation}

\begin{equation}0 \le P(x|c) P(c) \le P(x)\end{equation}
\begin{equation}0 \le P(c) \le \frac{P(x)}{P(x|c)}\end{equation}

\section{Experiment - Simple}

\note{This first experience is very small and was only used to better explain the effects of the threshold functions.}

For a better understanding of how the knowledge of the unconditional probability $P(x)$ can help to improve the detection threshold
a very small toy experiment was run:

The detection system is given access to labelled data ($known = \{kitchen, corridor\}$)
filtered from the distribution described in \autoref{tab:simple-distribution} as well to unlabelled samples from it.
From that input it learns both $P(x)$ and $P(x|class \in known)$.
Later the whole input space is sorted by the diferent thresholds for analysis.

Since the input space is very small (6 possible inputs), simple accumulation tables where used to learn both distributions.

\begin{table}[h]
\begin{center}
\input{simple-data/explain}
\end{center}
\caption{\label{tab:simple-distribution}Distribution used on the simple experiment.}
\end{table}

\begin{table}[h]
\begin{center}

\begin{tabular}{ |c| |c|}
\hline
$P(x|class \in know)$ & $\frac{P(x|class \in know)}{P(x)}$ \\ \hline
(kitchen, square) = 7.12 & (corridor, elongated) = 2 \\ \hline
(kitchen, square) = 7.12 & (corridor, elongated) = 2 \\ \hline
(kitchen, square) = 7.12 & (corridor, elongated) = 2 \\ \hline
(kitchen, square) = 7.12 & (corridor, elongated) = 2 \\ \hline
(kitchen, square) = 7.12 & (corridor, elongated) = 2 \\ \hline
\end{tabular}

\end{center}
\caption{\label{tab:simple-threshold}Input space sorted by threshold.}
\end{table}
\section{Experiment}
\label{sec:experiment}

\begin{itemize}
\item Define a set of features (all extractable from visual or user-tags).
\item Define a dataset of inputs (eg.: indoor environs at LabelMe)
\item Filter those for \emph{kitchen}.
\item Use ML methods to model $P(features | kitchen)$ and $P(features)$.
\item Test results.
\end{itemize}

\bibliographystyle{plainnat}
\bibliography{refs}

\end{document}

