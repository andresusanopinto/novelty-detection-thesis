% Additional packages
\usepackage{rotating}
\usepackage{hyphenat}
\usepackage[utf8]{inputenc}
\usepackage{subfig}
\usepackage[pdftex,pdfpagelabels,bookmarks,hyperindex,hyperfigures]{hyperref}
\hypersetup{%
   plainpages=false, 
   pdfpagelayout=SinglePage,
   bookmarksopen=false,
   bookmarksnumbered=true,
   breaklinks=true,
   linktocpage,
   colorlinks=true,
   linkcolor=blue,
   urlcolor=blue,
   citecolor=blue,
   anchorcolor=green
}      

% There seems to exist some kind of standard on using CHapter with capital.
\def\chapterautorefname{Chapter}
\newcommand{\subfigureautorefname}{\figureautorefname}

\newtheorem{algorithm}{Algorithm}

% Based on: http://lists.cs.princeton.edu/pipermail/topic-models/2010-December/001081.html
% http://www.mpi-inf.mpg.de/~dietz/probabilistic-models-tikz.zip
% by Laura Dietz (dietz at mpi-inf.mpg.de)
%
\usepackage{color}
\usepackage{array}
\usepackage{verbatim}
\usepackage{float}
\usepackage{amsmath}
\usepackage{amssymb}
\usepackage{esint}

%%%%%%%%%%%%%%%%%%%%%%%%%%%%%% LyX specific LaTeX commands.
%% Because html converters don't know tabularnewline
\providecommand{\tabularnewline}{\\}
\floatstyle{ruled}
\newfloat{algorithm}{tbp}{loa}[chapter]
\floatname{algorithm}{Algorithm}

%%%%%%%%%%%%%%%%%%%%%%%%%%%%%% Textclass specific LaTeX commands.
\usepackage{float}
\floatstyle{ruled}
\newfloat{algorithm}{tbp}{loa}
\floatname{algorithm}{Algorithm}
\newfloat{genmodel}{h}{loa}
\floatname{genmodel}{Generative Process}
\usepackage[noend]{algorithmic}
\newcommand{\forbody}[1]{ #1 \ENDFOR}
\newcommand{\ifbody}[1]{ #1  \ENDIF}
\newcommand{\whilebody}[1]{ #1  \ENDWHILE}
\renewcommand{\algorithmicprint}{\textbf{draw}}

%%%%%%%%%%%%%%%%%%%%%%%%%%%%%% User specified LaTeX commands.


\usepackage{euscript}

\DeclareSymbolFont{rsfscript}{OMS}{rsfs}{m}{n}
\DeclareSymbolFontAlphabet{\mathrsfs}{rsfscript}


% PDF formatting instructions for A4
\pdfpagewidth=210mm % for pdflatex
\pdfpageheight=296mm % for pdflatex


%%%%%%%% begin tikz %%%%%%
\usepackage{tikz,tkz-base}
\usetikzlibrary{shapes,decorations,shadows}
\usetikzlibrary{decorations.pathmorphing}
\usetikzlibrary{decorations.shapes}
\usetikzlibrary{fadings}
\usetikzlibrary{patterns}
\usetikzlibrary{calc}
\usetikzlibrary{decorations.text}
\usetikzlibrary{decorations.footprints}
\usetikzlibrary{decorations.fractals}
\usetikzlibrary{shapes.gates.logic.IEC}
\usetikzlibrary{shapes.gates.logic.US}
\usetikzlibrary{fit,chains}
\usetikzlibrary{positioning}
\usepgflibrary{shapes}
\usetikzlibrary{scopes}
\usetikzlibrary{arrows}
\usetikzlibrary{backgrounds}


\pgfdeclarelayer{background}
\pgfdeclarelayer{foreground}
\pgfsetlayers{background,main,foreground}

\tikzset{latent/.style={circle,fill=white,draw=red,thick,inner sep=1pt, 
minimum size=20pt, font=\fontsize{10}{10}\selectfont},
obs/.style={latent,fill=gray!25},
const/.style={rectangle, inner sep=0pt},
factor/.style={rectangle, fill=red,minimum size=7pt, inner sep=0pt},
yellow/.style={latent,minimum size=15pt,fill=yellow!75},
blue/.style={latent,minimum size=15pt,fill=blue!75},
-/.style={color=red, thick},
>={triangle 45}}




% shapename, fitlist, caption, pos
\newcommand{\plate}[4]{
\node (invis#1) [draw, transparent, inner sep=1pt,rectangle,fit=#2] {};
\node (capt#1) [ below left=0 pt of invis#1.south east, xshift=0pt,yshift=-9pt] {\raisebox{0pt}[0pt]{\footnotesize{#3}}};
\node (#1) [draw=black!50,thick,inner sep=3pt,rectangle,rounded corners,fit=(invis#1) (capt#1),#4] {};
}


\newcommand{\shiftedplate}[5]{
\node (invis#1) [draw, transparent, inner sep=0 pt,rectangle,fit=#2] {};
\node (capt#1) [#5, xshift=2pt] {\footnotesize{#3}};
\node (#1) [draw,inner sep=2pt, rectangle,fit=(invis#1) (capt#1),#4] {};
}

%shapename, pos, caption, in1, in2, out, captpos
\newcommand{\twofactor}[7]{
\node (#1) [factor] at #2 {};
\node (capt#1) [#7 of #1]{\footnotesize{#3}};
\draw [-] (#4) -- (#1) ;
\draw [-] (#5) -- (#1) ;
\draw [->,thick] (#1) -- (#6);
}

%shapename, pos, caption, in, out, captpos
\newcommand{\factor}[6]{
\node (#1) [factor] at #2 {};
\node (capt#1) [#6 of #1]{\footnotesize{#3}};
\draw [-] (#4) -- (#1) ;
\draw [->,thick] (#1) -- (#5);
}

% name, --, caption, pos
\newcommand{\nofactor}[4]{
\node (#1) [factor, #2]  {};
\node (capt#1) [#4 of #1]{\footnotesize{#3}};
}

%shapename,  fitlist, caption
\newcommand{\namedgate}[3]{
\node (invisgate#1) [rectangle, draw, transparent,  fit=#2] {};
\node (gatecapt#1) [ above right=0 pt of invisgate#1.north west, xshift=-1pt ] {\footnotesize{#3}};
\node (#1) [rectangle,draw,dashed, inner sep=2pt, fit=(invisgate#1)(gatecapt#1)]{};

}

%shapename,  fitlist, caption
\newcommand{\gate}[3]{
\node (#1) [rectangle,draw,dashed, inner sep=2pt, fit=#2]{};
}

%shapename,  fitlist1, fitlist2, caption1, caption2
\newcommand{\vertgate}[5]{
\node (invisgateleft#1) [rectangle, draw, transparent,  fit=#2] {};
\node (gatecaptleft#1) [ above left=0 pt of invisgateleft#1.north east, xshift=1pt ]{\footnotesize{#3}};
\node (invisgateright#1) [rectangle, draw, transparent,  fit=#4] {};
\node (gatecaptright#1) [ above right=0 pt of invisgateright#1.north west, xshift=-1pt ] {\footnotesize{#5}};
\node (#1) [rectangle,draw,dashed, inner sep=2pt, fit=(invisgateleft#1)(gatecaptleft#1)(invisgateright#1)(gatecaptright#1)]{};
\draw [-, dashed] (#1.north) -- (#1.south);
}


\newcommand{\vertgateSpec}[5]{
\node (invisgateleft#1) [rectangle, draw, transparent,  fit=#2] {};
\node (gatecaptleft#1) [ above left=0 pt of invisgateleft#1.north east, xshift=1pt ]{\footnotesize{#3}};
\node (invisgateright#1) [rectangle, draw, transparent,  fit=#4] {};
\node (gatecaptright#1) [ above right=0 pt of invisgateright#1.north west, xshift=-1pt ] {\footnotesize{#5}};
\node (#1) [rectangle,draw,dashed, inner sep=2pt, fit=(invisgateleft#1)(gatecaptleft#1)(invisgateright#1)(gatecaptright#1)]{};
\draw [-, dashed] (#1.70) -- (#1.290);
}

\newcommand{\horgate}[5]{
\node (invisgateleft#1) [rectangle, draw, transparent,  fit=#2] {};
\node (gatecaptleft#1) [ above right=0 pt of invisgateleft#1.south west, xshift=1pt ]{\footnotesize{#3}};
\node (invisgateright#1) [rectangle, draw, transparent,  fit=#4] {};
\node (gatecaptright#1) [ below right=0 pt of invisgateright#1.north west, xshift=-1pt ] {\footnotesize{#5}};
\node (#1) [rectangle,draw,dashed, inner sep=2pt, fit=(invisgateleft#1)(gatecaptleft#1)(invisgateright#1)(gatecaptright#1)]{};
\draw [-, dashed] (#1.west) -- (#1.east);
}

\newcommand{\horogate}[5]{
\node (invisgateleft#1) [rectangle, draw, transparent,  fit=#2] {};
\node (invisgateright#1) [rectangle, draw, transparent,  fit=#4] {};
\node (#1) [rectangle,draw,dashed, inner sep=2pt, fit=(invisgateleft#1)(invisgateright#1)]{};
\node (gatecaptleft#1) [ above right=0 pt of #1.west, xshift=0pt ]{\footnotesize{#3}};
\node (gatecaptright#1) [ below right=0 pt of #1.west, xshift=0pt ] {\footnotesize{#5}};

\draw [-, dashed] (#1.west) -- (#1.east);
}


\newcommand{\vertogate}[5]{
\node (invisgateleft#1) [rectangle, draw, transparent,  fit=#2] {};
\node (invisgateright#1) [rectangle, draw, transparent,  fit=#4] {};
\node (#1) [rectangle,draw,dashed, inner sep=2pt, fit=(invisgateleft#1)(invisgateright#1)]{};
\node (gatecaptleft#1) [ below left=0 pt of #1.north, xshift=0pt ]{\footnotesize{#3}};
\node (gatecaptright#1) [ below right=0 pt of #1.north, xshift=0pt ] {\footnotesize{#5}};

\draw [-, dashed] (#1.north) -- (#1.south);
}

