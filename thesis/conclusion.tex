\chapter{Conclusions and Future Work}\label{chap:conclusions}

% Summary of developed work
This thesis studied the problem of detecting novel situations where a robot lacks knowledge to
correctly describe them.
It did so on the area of semantic mapping on indoor spaces, by detecting that none of known room
categories was able to correctly explain the sensed properties.

For that it reviewed novelty detection and how an optimal detector can be implemented by
thresholding. Showing after that, with the assumption of constant probability of seeing a
novel case, an optimal ordering function for thresholding can be implemented based on
the factor between a conditional and unconditional probability.

It studied the semantic mapping process proposed by \cite{pronobis2011semmap} and presented a method
to detect novel room categories based on probabilistic graphical models.
Using a synthetic dataset, respecting the assumptions, it showed that such a method would be optimal
if unconditional probability could be optimally approximated.
Since in realistic conditions unconditional probability cannot be implemented due to the lack of
knowing all the classes, some approaches were performed to approximate it with either a uniform
assumption or by approximating with simplified models using unlabelled data.

On the rest of this chapter the main results and conclusions are presented.
Additionally limitations of the presented method and directions for future work are given.


\section{Results and Conclusions}

After studying the used semantic mapping process and studying novelty detection methods, this
thesis proposed modelling a conditional and an unconditional probability distribution of sensed
data using graphical models. Under the presented assumptions and assuming both can be approximated
with perfect accuracy a ratio between those both probabilities would yield an optimal ordering
function for implementing a novelty detector.

This case, was showed to be optimal using a synthetic dataset that simulates a simplified
environment of semantic categorization of room based on sense properties.
Being the correct way to model the conditional probability with the probabilistic graphical model
used by the semantic mapping process, this thesis moved forward with testing methods to approximate
the unconditional probability. It tested the usage of a uniform distribution and explored the usage
of unlabelled data to produce better models.

Additionally it was studied how the performance of the presented methods change as more data
is sensed from the environment. As expected all the methods increase their performance, but they
move further from what would optimally be expected.
Additionally the disadvantage between picking a uniform assumption and using an highly simplified
model for the unlabelled data, starts reducing as more information is available.
Nonetheless the model based on the usage of unlabelled data consistently lead to better detector
performance, being a strong indicator to use unlabelled data whenever there is access to it.

As a note, all the results presented on this thesis are directly reproducible from an
online\footnote{\url{https://github.com/andresusanopinto/novelty-detection-thesis}}
repository. That repository besides containing all the code and data for the results also
includes tech-reports, presentations, articles and notes that have been produced during
this thesis work. Additionally future research by the author will be correctly linked there,
when appropriate.

\section{Limitations}

The presented methods use a strong assumption on a constant $P(novel)$, this is unrealistic and
forbids one to exploit graph structural information that could show a given variable is more likely
to be unknown to the system because the given graph structure is not easily explained by the
current knowledge.
An example of that can be seen as: if all the agent knows are two categories that are very often
seen as a bi-colored graph, when presented with a non-bipartite graph, the agent should consider
the likelihood of a new category greater than on bipartite graphs.

Additionally, all the presented methods require calculation of $P(x|\overline{novel})$ and
$P(x)$. This way a full model of the graphical model is needed and uncertain sensing as described in
\autoref{sec:cues-from-low-level} cannot be easily incorporated.

This two limitations are a single symptom of the decision on inverting the conditional probability
$P(novel|x)$ to obtain a threshold that can be directly modelled by data.
It is expected that by trying to model how the potentials in the factor graphs change between the
conditional and unconditional graphical model, both limitations can be surpassed.

\section{Future Work}

Future work should try to study and or work around the limitations above presented. At the moment
it is not clear on how the threshold changes when the assumption on constant $P(novel)$ cannot be
made, and attempts to bypass the listed limitations will need to consider that by developing a
method where a static threshold can have a more realistic and controlled behaviour through any graph
structure.

At a longer and larger scale the following paragraphs describe possible and interesting directions
to exploit in the context of detecting knowledge gaps on artificial intelligence systems:


% \subsection*{Extend Novelty to All Variable Types on the Graph}
% Extend novelty to all categories on the graph (room, size, shape, etc..)
% \subsection*{Handle Uncertain and Novel Sensing from Low-level Classifiers}
% Handle uncertain sensing
\subsubsection*{Generalized Framework}
The presented method should be generalized by allowing novelty to be performed on any variable of
graphical model. Additionally the novelty information could be incorporated back in the graph
allowing the agent to probabilistically reason even when variables are considered unknown.
That generalization should aim at being fully probabilistic such as the system presented by
\cite{ranganathan2010pliss} and not deterministic by having to make decisions on which variables
it considers novel.

Additionally several methods exists that allow to produce novelty signals from the low-level
classifiers. A generalized framework should try to fuse and handle all that information by
incorporate it back in the graphical model in a similar fashion on how uncertain sensing is
performed.

% \subsection*{Explain Why a Certain Sample is Novel}
% Explain why a certain sample is novel
% \subsection*{Generation of most Likely Novel Samples}
% Generate most likely novel sample
\subsubsection*{Exploiting Generative Models}
An interesting aspect that arises from the use of generative models is the ability to generate new
samples according to it. It is expected that this can be exploited to achieve a better understanding
of what the system is modelling and what understand common properties between several novel samples
that allow to explain the novelty back to user. For example by generating the most
likely novel samples, it maybe be possible to understand the limitations of its
knowledge and use them for active learning.

% \subsection*{Probabilistic Modelling of Space Segmentation}
% \subsection*{Learn New Concepts}
% Learn Concepts / hidden variable types - (areas, rooms, floors, environments, etc\dots)
\subsubsection*{More than Novel Semantic Categories: Learning Graph Structures}
Though this thesis only touched the problem of detecting novel semantic categories, other knowledge
types should also be considered incomplete and are also candidates for novelty detection and
modelling tools that allow to handle the associated uncertainty on them.
More concretely, besides detecting novel semantic categories on given spatial concepts, there is
also interest on detecting novel concepts: the presence of hidden variables of a new type, not known
for the system: for example detecting room areas or types of environment.

For that an initial step should be made using probabilistic graph structures to model the several
possible space segmentations. Allowing to use a probabilistic approach instead of a deterministic
approach where space is segmented based on the presence of doors or other landmarks.

\subsubsection*{Beyond Detection of Knowledge Gaps}
Having methods to detect gaps of knowledge is just one of the first step on creating long-term and
life-long adaptable systems that are capable of learning through time. After detecting new
situations, ways to learn and incorporating the knowledge back on the agent must be developed.

