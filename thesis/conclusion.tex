\chapter{Conclusions and Future Work}\label{chap:conclusions}

% Summary of developed work
This thesis studied the problem of detecting novel situations in case of which the robot
lacks sufficient knowledge to explain the sensory information. It did so in the area of 
semantic mapping on indoor spaces, and more precisely, for the problem of detecting the situation when
none of the known room category models were able to correctly explain the sensed spatial properties.

It reviewed novelty detection and how an optimal detector can be implemented by
thresholding. It also showed that with the assumption of constant probability of seeing a
novel case, an optimal ordering function for thresholding can be implemented based on
the factor between conditional and unconditional probability.

It studied the semantic mapping process proposed in \cite{pronobis2011semmap} and presented 
a method to detect novel room categories based on probabilistic graphical models.
Using a synthetic dataset, respecting the assumptions, it showed that such a method would be 
optimal if unconditional probability could be optimally approximated. Since, in realistic conditions, 
unconditional probability cannot be obtained due to the lack of knowledge about all possible classes, various
techniques were used to approximate it with either a uniform assumption or simplified models built from 
unlabelled data.

The rest of this chapter presents the main results and conclusions. Additionally limitations of the 
presented methods and directions for future work are discussed.


\section{Results and Conclusions}

After studying the semantic mapping process and novelty detection methods, this
thesis proposed modelling the conditional and unconditional probability distribution of sensed
data using graphical models. Under the presented assumptions, and assuming both distributions can be 
exactly known, a ratio between those probabilities would yield an optimal ordering function for implementing a
novelty detector. This case was showed to be optimal using a synthetic dataset that simulates a simplified
environment for semantic categorization of rooms based on sensed properties.
As the next step, this thesis tested methods to approximate the unconditional probability. 
It tested the usage of a uniform distribution and explored the usage of unlabelled data to 
produce better models.

Additionally it studied how the performance of the presented methods changed as more data
was sensed from the environment. As expected, all the methods increase their performance, but they
move further away from what could be obtained with exact information.
Additionally, the disadvantage of the uniform assumption compared to using a highly simplified
model for the unlabelled data starts reducing as more information is available.
Nonetheless the model based on the usage of unlabelled data consistently lead to better detector
performance, which is a strong indication that unlabelled data should be used whenever it is available.

As a note, all the results presented in this thesis are directly reproducible from an
online\footnote{\url{https://github.com/andresusanopinto/novelty-detection-thesis}}
repository. That repository besides containing all the code and data for the results also
includes tech-reports, presentations, articles and notes that have been produced during
this thesis work. Additionally, future research by the author will be correctly linked there,
when appropriate.


\section{Limitations}

The presented methods use a strong assumption on a constant $P(novel)$, this is unrealistic and
forbids one to exploit graph structural information that could show a given variable is more likely
to be unknown to the system because the given graph structure is not easily explained by the
current knowledge. As as example of that, consider the following: if all the agent knows are two 
categories that are very often seen as a bi-colored graph, when presented with a non-bipartite graph, 
the agent should consider the likelihood of a new category greater than on bipartite graphs.

Additionally, all the presented methods require calculation of $P(x|\overline{novel})$ and
$P(x)$. This way a full graphical model is needed and uncertain sensing as described in
\autoref{sec:cues-from-low-level} cannot be easily incorporated.

These two limitations are a single symptom of the decision on inverting the conditional probability
$P(novel|x)$ to obtain a threshold that can be directly modelled by the data.
It is expected that by trying to model how the potentials in the factor graphs change between the
conditional and unconditional graphical model, both limitations can be surpassed.


\section{Future Work}

Future work will try to study and work around the limitations presented above. At the moment
it is not clear how the threshold changes when the assumption about constant $P(novel)$ cannot be
made. Any attempts to bypass the listed limitations will need to consider that problem by 
developing a method where a threshold can have a more realistic and controlled behaviour 
for any graph structure.

The following paragraphs describe possible and interesting directions to explore in the context 
of detecting knowledge gaps in artificial intelligent systems:

% \subsection*{Extend Novelty to All Variable Types on the Graph}
% Extend novelty to all categories on the graph (room, size, shape, etc..)
% \subsection*{Handle Uncertain and Novel Sensing from Low-level Classifiers}
% Handle uncertain sensing
\subsubsection*{Generalized the Framework}
The presented method should be generalized by allowing novelty detection to be performed on any 
variable of the graphical model. Additionally, the novelty information could be incorporated back 
into the graph allowing the agent to probabilistically reason even when variables are considered 
unknown. That generalization should aim at being fully probabilistic such as the system presented by
\cite{ranganathan2010pliss} and not deterministic by having to make decisions on which variables
it considers novel.

Additionally, several methods exists that allow to produce novelty signals from the low-level
classifiers. A generalized framework should try to fuse and handle all that information by
incorporating it into the graphical model in a similar fashion as how uncertain sensing outcomes are incorporated.

% \subsection*{Explain Why a Certain Sample is Novel}
% Explain why a certain sample is novel
% \subsection*{Generation of most Likely Novel Samples}
% Generate most likely novel sample
\subsubsection*{Exploiting Generative Models}
An interesting aspect that arises from the use of generative models is the ability to generate new
samples from the models. It is expected that this can be exploited to achieve better understanding
of what the system is modelling and what explain the character of the novelty back to the user.
Moreover, this feature can be used to refine and actively update the models by confronting the captured 
knowledge with the human understanding of the modeled concepts.

% \subsection*{Probabilistic Modelling of Space Segmentation}
% \subsection*{Learn New Concepts}
% Learn Concepts / hidden variable types - (areas, rooms, floors, environments, etc\dots)
\subsubsection*{Beyond Novel Semantic Categories: Learning Graph Structures}
Although this thesis only touched the problem of detecting novel semantic categories, other knowledge
types should also be considered incomplete and are also candidates for novelty detection.
More concretely, besides detecting novel semantic categories for given spatial concepts, there is
also interest in detecting novel concepts: the presence of hidden variables of a new type, not known
to the system: for example detecting rooms or types of environment.

An initial step towards that goal could be probabilistic learning of graph structures to model various
possible space segmentations. This would permit the use of a probabilistic approach instead of the currently used
deterministic approach where space is segmented based on the presence of doors or other landmarks.

\subsubsection*{Beyond Detection of Knowledge Gaps}
Having methods to detect gaps of knowledge is just one of the first steps towards creating long-term and
life-long adaptable systems that are capable of learning over time. After detecting new
situations, ways to learn and incorporate the knowledge into the models must be developed.

