\chapter{Conclusions and Future Work}\label{chap:conclusions}

% Summary of developed work
This thesis studied the problem of detecting situations where the knowledge of the robot
is not 
Gaps those, that disable the robot to function 
It did so, specializing 
that allows novelty detection of semantic room
categories using as base the semantic mapping process proposed by \cite{pronobis2011semmap}.


\url{https://github.com/andresusanopinto/novelty-detection-thesis}

On the rest of this chapter the main results and conclusions are presented.
Additionally limitations of the presented method and directions for future work are given.



\section{Results and Conclusions}

\section{Limitations}

\begin{itemize}
\item Strong assumption on constant $P(novel)$
\item Threshold should actually be dynamic with structure
\item Unable to drop normalizations factors
\item Handle multiple novel variables at the same time
\end{itemize}
 

\section{Future Work}

\begin{itemize}
\item Extend novelty to all categories on the graph (room, size, shape, etc..)
\item Handle uncertain sensing
\item Probabilistic Space Segmentation (learn how to structure the graph)
\item Learn Concepts / hidden variable types - (areas, rooms, floors, environments, etc\dots)
\item Explain why a certain sample is novel
\item Generate most likely novel sample
\end{itemize}
