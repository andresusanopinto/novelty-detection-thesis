% FEUP THESIS STYLE for LaTeX2e
% how to use feupteses (changed from the original for MIEEC)
%
% FEUP, JCL & JCF, Tue May 20 18:53:15 2008
%
% PLEASE send improvements to jlopes at fe.up.pt, jcf at fe.up.pt
%

%%========================================
%% Commands: pdflatex mieic
%%           bibtex mieic
%%           makeindex mieic (only if crating an index) 
%%           pdflatex mieic
%%========================================

%% For one side layout comment next line and uncomment the second line
\documentclass[11pt,a4paper,twoside,openright]{report}
%\documentclass[11pt,a4paper]{report}

%% For iso-8859-1 (latin1), comment next line and uncomment the second line
\usepackage[utf8]{inputenc}
%\usepackage[latin1]{inputenc}

%% Use option portuges if needed
\usepackage[english]{babel}

%% For the final version, comment next line and uncomment the second line
\usepackage[provisional,alpharefs]{feupteses}      
%\usepackage[alpharefs]{feupteses} 

%% Options: 
%% - portuges: titles, etc in portuguese
%% - provisional: the thesis has not been approved yet
%% - usewatermark: use watermark instaed of provisonal text
%% - print: links are not shown (for paper versions)
%% - alpharefs: bibliography references are alphabetic
%% - numericrefs: bibliography references are numbered (in order of citation)
%% ( by default: author-date format of the ``natbib'' package is used 
%%   the portuguese version requires the file ``plainnat-pt.bst'' to be 
%%   present in the same directory )

%% Include MIEIC definitions different from standard style
\usepackage{mieicpatch}

%% Provide a version number in order to keep track of
%% thesis versions (it will printed in the footer of most pages)

\version{0.99}

%% Uncomment in the final version in order to make version footer disappear
\noversiontrue                 

%% Uncomment to create an index (at the end of the document)
%\makeindex                      

%% Path to the figures directory
%% TIP: use folder ``figures'' to keep all your figures
% \graphicspath{{figures/}}       

% For making gloassaries.
% http://en.wikibooks.org/wiki/LaTeX/Glossary
\usepackage[acronym]{glossaries}
\makeglossaries

%%----------------------------------------
%% TIP: if you want to define more macros, use an external file to
%% keep them
% Additional packages
\usepackage[utf8]{inputenc}
\usepackage{subfig}
\usepackage[pdftex,pdfpagelabels,bookmarks,hyperindex,hyperfigures]{hyperref}
\hypersetup{%
   plainpages=false, 
   pdfpagelayout=SinglePage,
   bookmarksopen=false,
   bookmarksnumbered=true,
   breaklinks=true,
   linktocpage,
   colorlinks=true,
   linkcolor=blue,
   urlcolor=blue,
   citecolor=blue,
   anchorcolor=green
}      

% There seems to exist some kind of standard on using CHapter with capital.
\def\chapterautorefname{Chapter}

\newtheorem{algorithm}{Algorithm}

% Based on: http://lists.cs.princeton.edu/pipermail/topic-models/2010-December/001081.html
% http://www.mpi-inf.mpg.de/~dietz/probabilistic-models-tikz.zip
% by Laura Dietz (dietz at mpi-inf.mpg.de)
%
\usepackage{color}
\usepackage{array}
\usepackage{verbatim}
\usepackage{float}
\usepackage{amsmath}
\usepackage{amssymb}
\usepackage{esint}

%%%%%%%%%%%%%%%%%%%%%%%%%%%%%% LyX specific LaTeX commands.
%% Because html converters don't know tabularnewline
\providecommand{\tabularnewline}{\\}
\floatstyle{ruled}
\newfloat{algorithm}{tbp}{loa}[chapter]
\floatname{algorithm}{Algorithm}

%%%%%%%%%%%%%%%%%%%%%%%%%%%%%% Textclass specific LaTeX commands.
\usepackage{float}
\floatstyle{ruled}
\newfloat{algorithm}{tbp}{loa}
\floatname{algorithm}{Algorithm}
\newfloat{genmodel}{h}{loa}
\floatname{genmodel}{Generative Process}
\usepackage[noend]{algorithmic}
\newcommand{\forbody}[1]{ #1 \ENDFOR}
\newcommand{\ifbody}[1]{ #1  \ENDIF}
\newcommand{\whilebody}[1]{ #1  \ENDWHILE}
\renewcommand{\algorithmicprint}{\textbf{draw}}

%%%%%%%%%%%%%%%%%%%%%%%%%%%%%% User specified LaTeX commands.


\usepackage{euscript}

\DeclareSymbolFont{rsfscript}{OMS}{rsfs}{m}{n}
\DeclareSymbolFontAlphabet{\mathrsfs}{rsfscript}


% PDF formatting instructions for A4
\pdfpagewidth=210mm % for pdflatex
\pdfpageheight=296mm % for pdflatex


%%%%%%%% begin tikz %%%%%%
\usepackage{tikz,tkz-base}
\usetikzlibrary{shapes,decorations,shadows}
\usetikzlibrary{decorations.pathmorphing}
\usetikzlibrary{decorations.shapes}
\usetikzlibrary{fadings}
\usetikzlibrary{patterns}
\usetikzlibrary{calc}
\usetikzlibrary{decorations.text}
\usetikzlibrary{decorations.footprints}
\usetikzlibrary{decorations.fractals}
\usetikzlibrary{shapes.gates.logic.IEC}
\usetikzlibrary{shapes.gates.logic.US}
\usetikzlibrary{fit,chains}
\usetikzlibrary{positioning}
\usepgflibrary{shapes}
\usetikzlibrary{scopes}
\usetikzlibrary{arrows}
\usetikzlibrary{backgrounds}


\pgfdeclarelayer{background}
\pgfdeclarelayer{foreground}
\pgfsetlayers{background,main,foreground}

\tikzset{latent/.style={circle,fill=white,draw=red,thick,inner sep=1pt, 
minimum size=20pt, font=\fontsize{10}{10}\selectfont},
obs/.style={latent,fill=gray!25},
const/.style={rectangle, inner sep=0pt},
factor/.style={rectangle, fill=red,minimum size=7pt, inner sep=0pt},
yellow/.style={latent,minimum size=15pt,fill=yellow!75},
blue/.style={latent,minimum size=15pt,fill=blue!75},
-/.style={color=red, thick},
>={triangle 45}}




% shapename, fitlist, caption, pos
\newcommand{\plate}[4]{
\node (invis#1) [draw, transparent, inner sep=1pt,rectangle,fit=#2] {};
\node (capt#1) [ below left=0 pt of invis#1.south east, xshift=0pt,yshift=-9pt] {\raisebox{0pt}[0pt]{\footnotesize{#3}}};
\node (#1) [draw=black!50,thick,inner sep=3pt,rectangle,rounded corners,fit=(invis#1) (capt#1),#4] {};
}


\newcommand{\shiftedplate}[5]{
\node (invis#1) [draw, transparent, inner sep=0 pt,rectangle,fit=#2] {};
\node (capt#1) [#5, xshift=2pt] {\footnotesize{#3}};
\node (#1) [draw,inner sep=2pt, rectangle,fit=(invis#1) (capt#1),#4] {};
}

%shapename, pos, caption, in1, in2, out, captpos
\newcommand{\twofactor}[7]{
\node (#1) [factor] at #2 {};
\node (capt#1) [#7 of #1]{\footnotesize{#3}};
\draw [-] (#4) -- (#1) ;
\draw [-] (#5) -- (#1) ;
\draw [->,thick] (#1) -- (#6);
}

%shapename, pos, caption, in, out, captpos
\newcommand{\factor}[6]{
\node (#1) [factor] at #2 {};
\node (capt#1) [#6 of #1]{\footnotesize{#3}};
\draw [-] (#4) -- (#1) ;
\draw [->,thick] (#1) -- (#5);
}

% name, --, caption, pos
\newcommand{\nofactor}[4]{
\node (#1) [factor, #2]  {};
\node (capt#1) [#4 of #1]{\footnotesize{#3}};
}

%shapename,  fitlist, caption
\newcommand{\namedgate}[3]{
\node (invisgate#1) [rectangle, draw, transparent,  fit=#2] {};
\node (gatecapt#1) [ above right=0 pt of invisgate#1.north west, xshift=-1pt ] {\footnotesize{#3}};
\node (#1) [rectangle,draw,dashed, inner sep=2pt, fit=(invisgate#1)(gatecapt#1)]{};

}

%shapename,  fitlist, caption
\newcommand{\gate}[3]{
\node (#1) [rectangle,draw,dashed, inner sep=2pt, fit=#2]{};
}

%shapename,  fitlist1, fitlist2, caption1, caption2
\newcommand{\vertgate}[5]{
\node (invisgateleft#1) [rectangle, draw, transparent,  fit=#2] {};
\node (gatecaptleft#1) [ above left=0 pt of invisgateleft#1.north east, xshift=1pt ]{\footnotesize{#3}};
\node (invisgateright#1) [rectangle, draw, transparent,  fit=#4] {};
\node (gatecaptright#1) [ above right=0 pt of invisgateright#1.north west, xshift=-1pt ] {\footnotesize{#5}};
\node (#1) [rectangle,draw,dashed, inner sep=2pt, fit=(invisgateleft#1)(gatecaptleft#1)(invisgateright#1)(gatecaptright#1)]{};
\draw [-, dashed] (#1.north) -- (#1.south);
}


\newcommand{\vertgateSpec}[5]{
\node (invisgateleft#1) [rectangle, draw, transparent,  fit=#2] {};
\node (gatecaptleft#1) [ above left=0 pt of invisgateleft#1.north east, xshift=1pt ]{\footnotesize{#3}};
\node (invisgateright#1) [rectangle, draw, transparent,  fit=#4] {};
\node (gatecaptright#1) [ above right=0 pt of invisgateright#1.north west, xshift=-1pt ] {\footnotesize{#5}};
\node (#1) [rectangle,draw,dashed, inner sep=2pt, fit=(invisgateleft#1)(gatecaptleft#1)(invisgateright#1)(gatecaptright#1)]{};
\draw [-, dashed] (#1.70) -- (#1.290);
}

\newcommand{\horgate}[5]{
\node (invisgateleft#1) [rectangle, draw, transparent,  fit=#2] {};
\node (gatecaptleft#1) [ above right=0 pt of invisgateleft#1.south west, xshift=1pt ]{\footnotesize{#3}};
\node (invisgateright#1) [rectangle, draw, transparent,  fit=#4] {};
\node (gatecaptright#1) [ below right=0 pt of invisgateright#1.north west, xshift=-1pt ] {\footnotesize{#5}};
\node (#1) [rectangle,draw,dashed, inner sep=2pt, fit=(invisgateleft#1)(gatecaptleft#1)(invisgateright#1)(gatecaptright#1)]{};
\draw [-, dashed] (#1.west) -- (#1.east);
}

\newcommand{\horogate}[5]{
\node (invisgateleft#1) [rectangle, draw, transparent,  fit=#2] {};
\node (invisgateright#1) [rectangle, draw, transparent,  fit=#4] {};
\node (#1) [rectangle,draw,dashed, inner sep=2pt, fit=(invisgateleft#1)(invisgateright#1)]{};
\node (gatecaptleft#1) [ above right=0 pt of #1.west, xshift=0pt ]{\footnotesize{#3}};
\node (gatecaptright#1) [ below right=0 pt of #1.west, xshift=0pt ] {\footnotesize{#5}};

\draw [-, dashed] (#1.west) -- (#1.east);
}


\newcommand{\vertogate}[5]{
\node (invisgateleft#1) [rectangle, draw, transparent,  fit=#2] {};
\node (invisgateright#1) [rectangle, draw, transparent,  fit=#4] {};
\node (#1) [rectangle,draw,dashed, inner sep=2pt, fit=(invisgateleft#1)(invisgateright#1)]{};
\node (gatecaptleft#1) [ below left=0 pt of #1.north, xshift=0pt ]{\footnotesize{#3}};
\node (gatecaptright#1) [ below right=0 pt of #1.north, xshift=0pt ] {\footnotesize{#5}};

\draw [-, dashed] (#1.north) -- (#1.south);
}


%%----------------------------------------

%%========================================
%% Start of document
%%========================================
\begin{document}

%%----------------------------------------
%% Information about the work
%%----------------------------------------
\title{Novelty Detection for Semantic Place Categorization}
\author{André Susano Pinto}
\degree{Master in Informatics and Computing Engineering}
%% Date of submission
\thesisdate{17$^{th}$ June, 2011}

%% Insert copyright text if used
%\copyrightnotice{Name of the Author, 2008}

\supervisor{Supervisor}{Luís Paulo Reis}{(Dr.)}
\supervisor{Second Supervisor}{Andrzej Pronobis}{(Dr.)}

%% Uncomment committee stuff in the final version
\committeetext{Approved in oral examination by the committee:}
\committeemember{Chair}{Name of the President}{(Title)}
\committeemember{External Examiner}{Name of the Examiner}{(Title)}
\committeemember{Supervisor}{Name of the Supervisor}{(Title)}
\signature
\committeedate{31$^{st}$ July, 2010}

%% Specify cover logo (in folder ``figures'')
\logo{figures/feup-logo.pdf}

%%----------------------------------------
%% Cover page(s)
%%----------------------------------------
\maketitle

%% Uncomment next line in the final version
\committeepage
%MIEEC uses an external PDF page with the signatures (juri.pdf)
%\includepdf[pagecommand={},noautoscale=false,fitpaper=true,pages=-]{juri.pdf}

%% Preliminary materials
\StartPrelim
\begin{singlespace}
  \chapter*{Abstract}

This should probably be written as last thing.

\chapter*{Resumo}
Let $E$ denote the set of all possible texts in English and $P$ the set of
all possible texts in Portuguese. A transformation $T_{E->P}(x)$ can be defined
in $E \times P$, such that it maps $x \in E$ to a translated version $y \in P$ such
that they hold the same semantic meaning to a reader.

Assume this text to be $T_{E->P}(abstract)$.
\footnote{What I mean is that this will be a translation of the english text present
on abstract and don't expect it to be filled in before the final version of this
document!}

 % the abstract
  \chapter*{Acknowledgements}
This thesis would have not been possible without the help of some other persons.
I would like to thank to Andrzej Pronobis, for introducing me to the problem and
directing it towards my interests in computer science, for the meetings, insights
and endless reviews of my work. Also to Carl Henrik Ek, for enthusiastically taking part
on the meetings discussing graphical models and related work.

The Computer Vision and Active Perception Lab at Royal Institute of Technology
in Stockholm were also a key piece for providing such a staff and facilities that
surrounded me during one year of thesis and exchange studies, and whose
courses replenish my interests in machine learning.
Also Faculdade de Engenharia of University of Porto who made it possible
for me to study and perform my thesis abroad.
Thanks to Luis Paulo Reis, for accepting to supervise my thesis and pushing me to submit a paper.

Virgile Högman, for besides handling me as a work colleague and chatter-box, becoming a friend.
And Erik Ass for all the interesting coding and math discussions on totally random problems.
I cannot also avoid to thank my neighbours back in Nockeby, in special to Diogo Gonçalves and Elise Löbker
that kept chatting with me through the nights during my thesis work.

Thanks to Mariatorget people who kept asking how my thesis was going and when it was going to be finished.
And for all the parties, drinking, relaxing and nights out! Thanks to Elina Säfsten, for being a cool neighbour
with whom I had the pleasure to spent quite some time talking and partying with her and her Swedish friends.
Thanks to Manuel Gattermayr and Tommaso Facchini for being awesome neighbours as well.

I am also grateful to Bernhard Schwaighofer, Manuel, Tommaso and others for an amazing
road trip, night fires, sunsets and sunrises on Sweden. Special thanks to Bernie for during the road trip
promising to wake me up before my sleeping bag starts to burn and melt with my skin and eventually turning me water proof.

And thanks to all the party people, not forgetting Andrzej, who showed me how to do a PhD party in case I ever decide
to seek one.

Last but not the least, thanks to my family and friends back in Porto and other parts of the world,
who kept talking and cheered me up.

% PS: This list does not contains all the persons the author of this thesis would like to thank, and the author should not be held liable for it.

\vspace{10mm}
\flushleft{
"I am thankful to all the awesome people who were part of this Stockholm chapter of my life"
--
\emph{André Susano Pinto}}

  % the acknowledgments
  %\cleardoublepage
\thispagestyle{plain}

\vspace*{8cm}

\begin{flushright}
   \textsl{``You should be glad that bridge fell down. \\
           I was planning to build thirteen more to that same design''} \\
\vspace*{1.5cm}
           Isambard Kingdom Brunel
\end{flushright}
    % initial quotation if desired
  \cleardoublepage
  \pdfbookmark[0]{Table of Contents}{contents}
  \tableofcontents
  \cleardoublepage
  \pdfbookmark[0]{List of Figures}{figures}
  \listoffigures
  \cleardoublepage
  \pdfbookmark[0]{List of Tables}{tables}
  \listoftables
  \cleardoublepage
  \pdfbookmark[0]{Glossary}{glossary}
  \printglossaries
  \cleardoublepage
  %\pdfbookmark[0]{Abbreviations}{abbrevs}
  %\chapter*{Abbreviations}
\chaptermark{ABBREVIATIONS}

\begin{flushleft}
\begin{tabular}{l p{0.8\linewidth}}
ADT      & Abstract Data Type\\
ANDF     & Architecture-Neutral Distribution Format\\
API      & Application Programming Interface\\
CAD      & Computer-Aided Design\\
CASE     & Computer-Aided Software Engineering\\
CORBA    & Common Object Request Broker Architecture\\
UNCOL    & UNiversal COmpiler-oriented Language\\
Loren    & Lorem ipsum dolor sit amet, consectetuer adipiscing
elit. Sed vehicula lorem commodo dui\\
WWW      & \emph{World Wide Web}
\end{tabular}
\end{flushleft}

  % the list of abbreviations used
\end{singlespace}

\newacronym{SVM}{SVM}{Support Vector Machine}
\newacronym{Dora}{Dora}{Dora: The Explorer}
\newacronym{CRFH}{CRFH}{Composed Receptive Field Histogram}
\newacronym{SIFT}{SIFT}{Scalar Invariant Feature Transform}
\newacronym{PCA}{PCA}{Principal Component Analysis}
\newacronym{K-PCA}{K-PCA}{Kernel Principal Component Analysis}
\newacronym{COLD}{COLD}{COsy Location Database}
\newacronym{MAP}{MAP}{Maximum a Posteriori}
\newacronym{KTH}{KTH}{Kungliga Tekniska Högskolan}
\newacronym{FEUP}{FEUP}{Faculdade de Engenharia da Universidade do Porto}
\newacronym{PLISS}{PLISS}{Place Labeling through Image Sequence Segmentation}


\newglossaryentry{ImageCLEF}
{
  name={ImageCLEF},
  description={is the cross-language image retrieval track which is run as part of the Cross Language Evaluation Forum}
}

\newglossaryentry{Kinect}
{
  name={Kinect},
  description={is a motion sensor introduced by Microsoft to create a controller-free gaming and entertainment experience. Since then it has been used by several researchers and hobbyist in the area of robotics}
}


%%----------------------------------------
%% Body
%%----------------------------------------
\StartBody

%% TIP: use a separate file for each chapter
\chapter{Introduction}
\label{chap:introduction}

%\section*{}
%This chapter gives a generic overview of the problem, its motivation and goals. It also describes how the rest of the document is organized.

\section{Motivation}
There has been several efforts in the area of Artificial Intelligence and Robotics in creating robots that are able to interact with humans and their environments.
One of the existing problems is a reliable high-level localization method that can be deployed into new and unknown environments.

That task is specially difficult due to the constant change in those environments, either introduced by human interaction or by other external factors such as light-conditions.
Also the generalization requisite on such task requires highly generic and stable features to be extracted.

This thesis will focus on man-made indoor environments such as houses, offices, labs.
Where it would be desirable to map robot position to an high-level description such as kitchen, corridor, printer-area, office.
Such a classification can then be used to:
\footnote{Besides the motivation scenario, visual place classification has uses on other areas like augmented reality, content-base image retrieval and context awareness~\citep{dey2000towards}.}
\begin{itemize*}
\item Improve human interaction by mapping the robot localization to human concepts.
\item Improve robot localization methods with a high-level and robust localization information.
\item Extend knowledge about room categories and their properties.
\item Provide the robot with the ability to perform context-aware decisions.
\end{itemize*}

The robots should be able to operate in unknown environments as often they cannot be trained on the same environment they will operate on.
And under those circumstances the ability to distinguish between the known and unknown becomes a key point for reliability since it allows a robot to not trust the results it gets on new types of rooms.

It is therefore important to develop and access the quality of methods to identify novel cases.
Being the detection of novelty a key point for several tasks such as:
\begin{itemize*}
\item Operation in unknown environments.
\item Modelling what is know.
\item Ability to self-extend knowledge.
\end{itemize*}

\subsection{Visual features}
\label{sec:visual_motivation}
% Why Visual Place Classification!!
A robot often has several sensors that capture characteristics of its surrounding environment.
From those, vision is the most interesting and rich one and nowadays it is very easy to incorporate.
Making it a primary source of information for place classification.

Although its also the richness of the vision sensors that make it noisy and harder do interpret as the appearance of places varies over time due to illumination, human activity and view change.
It becomes then important to extract stable features from the visual input.
Visual features will be the main features explored during the thesis although other methods will also be used.

\section{Related Work}
\label{sec:related-work}
\cite{quattoni2009recognizing} showed that most scene recognition models work poorly in indoor scenes when compared to outdoor scenes results.
Since the properties that characterize rooms changes conforming its category. Namely corridors are well described by global properties and bookstores are well described by the presence of specific objects (books).
It became obvious then to use information provided by several sources. Their work uses both global and local features for scene recognition and does not address any specific information available in the context of mobile robotics.

This relation between room category and object has also been studied in the object search field.
Object search mainly focus on geometric properties but \cite{galindo2005multi} defines a bidirectional relation between object and room category, where object defines a room category and a room category provides information on where objects may be found.

Probabilistic representations are used in several localised functions in robots operating in the real-world~\citep{gross2009toomas,maierprobabilistic}. And some employ, up to some extent, a probabilistic representation across some subsystems~\citep{kraft2008exploration}.
\citet{vasudevan2008bayesian} performed room categorization through Bayesian reasoning about the presence of objects but did not included observations models (perception is considered deterministic).
And \cite{boutell2006factor} have studied outdoor scene classification using \emph{factor graphs} and modelling spatial relations between objects in the scene to extract better knowledge from semantic (high-level) features.

Its expected that using a unified probabilistic model from the whole system, such as \cite{pronobis2011exploiting}, more information can be reused to correctly predict a given random variable.

While there has been active research on visual properties and place classification, novelty detection applied to this problem has not seen much work on it~\citep{caputo2009overview}.
As \cite{markou2003novelty} reviews, novelty detection is an incredibly complex problem and requires specific techniques and methods to each problem.

\cite{bishop1994novelty} has showed that unconditional probability density can be use to provide a novelty measure. Though that probability is in most cases impractical to measure and even in those cases its necessary to find a correct threshold for \emph{novelty detection}. In higher-dimensions this method loses precision due to a spread out of the probability density function as most probability will be spread out on the tails of the function~\citep{markou2003novelty-part2}.

For that reason several other approaches have been developed for novelty detection.
One of those is the work of \cite{Hoffmann2007863} which applied non-linear statistical analysis to detect novel cases.
It has shown good results when applied to several problems such as digit recognition and cancer detection.

This type of approach although suffers from high computational and memory needs and often techniques need to be adapted to allow a online behaviour~\citep{sofman2010anytime}.


\section{Contribution/Goals}
\label{sec:goals}
During the thesis a thorough evaluation of a recently proposed property-based semantic mapping system for mobile robots~\citep{pronobis2011exploiting} on a real world visual database will be performed.

That system will be extended by usage of state-of-art novelty detection machine learning algorithms to the problem of visual place categorization.
As a final evaluation step, the developed method will be submitted to the Robot Vision Task on \gls{ImageCLEF}.

\section{Outline}
The rest of this technical report is organized as follows:

\begin{description}
\item[Chapter \ref{chap:background}] introduces the background for handling the presented problem.
It introduces the generic classification problem and techniques used to address it. Later developing on the specific visual features normally used as input for the place classification.

It also presents the novelty detection problem as well the techniques used to address it in the current context.

\item[Chapter \ref{chap:approach}] presents our approach to the problem. It introduces the platform over which the visual place classification is performed and points in the direction of extending such a platform for novelty detection.

\item[Chapter \ref{chap:testing}] introduces the testing and evaluation methodologies.
It also presents the databases used and the \gls{ImageCLEF} competition to which the developed work will be submitted.

\item[Chapter \ref{chap:workplan}] lists and elaborates on the planned tasks to be completed during the master thesis work and gives an estimated schedule for the working time. 
\end{description}

 
\chapter{Results}
\section*{}

Report experiments and results.

\chapter{Conclusions and Future Work}\label{chap:conclusions}

% Summary of developed work
This thesis studied the problem of detecting novel situations where a robot lacks knowledge to
correctly describe them.
It did so on the area of semantic mapping on indoor spaces, by detecting that none of known room
categories was able to correctly explain the sensed properties.

For that it reviewed novelty detection and how an optimal detector can be implemented by
thresholding. Showing after that, with the assumption of constant probability of seeing a
novel case, an optimal ordering function for thresholding can be implemented based on
the factor between a conditional and unconditional probability.

It studied the semantic mapping process proposed by \cite{pronobis2011semmap} and presented a method
to detect novel room categories based on probabilistic graphical models.
Using a synthetic dataset, respecting the assumptions, it showed that such a method would be optimal
if unconditional probability could be optimally approximated.
Since in realistic conditions unconditional probability cannot be implemented due to the lack of
knowing all the classes, some approaches were performed to approximate it with either a uniform
assumption or by approximating with simplified models using unlabelled data.

On the rest of this chapter the main results and conclusions are presented.
Additionally limitations of the presented method and directions for future work are given.


\section{Results and Conclusions}

After studying the used semantic mapping process and studying novelty detection methods, this
thesis proposed modelling a conditional and an unconditional probability distribution of sensed
data using graphical models. Under the presented assumptions and assuming both can be approximated
with perfect accuracy a ratio between those both probabilities would yield an optimal ordering
function for implementing a novelty detector.

This case, was showed to be optimal using a synthetic dataset that simulates a simplified
environment of semantic categorization of room based on sense properties.
Being the correct way to model the conditional probability with the probabilistic graphical model
used by the semantic mapping process, this thesis moved forward with testing methods to approximate
the unconditional probability. It tested the usage of a uniform distribution and explored the usage
of unlabelled data to produce better models.

Additionally it was studied how the performance of the presented methods change as more data
is sensed from the environment. As expected all the methods increase their performance, but they
move further from what would optimally be expected.
Additionally the disadvantage between picking a uniform assumption and using an highly simplified
model for the unlabelled data, starts reducing as more information is available.
Nonetheless the model based on the usage of unlabelled data consistently lead to better detector
performance, being a strong indicator to use unlabelled data whenever there is access to it.

As a note, all the results presented on this thesis are directly reproducible from an
online\footnote{\url{https://github.com/andresusanopinto/novelty-detection-thesis}}
repository. That repository besides containing all the code and data for the results also
includes tech-reports, presentations, articles and notes that have been produced during
this thesis work. Additionally future research by the author will be correctly linked there,
when appropriate.

\section{Limitations}

The presented methods use a strong assumption on a constant $P(novel)$, this is unrealistic and
forbids one to exploit graph structural information that could show a given variable is more likely
to be unknown to the system because the given graph structure is not easily explained by the
current knowledge.
An example of that can be seen as: if all the agent knows are two categories that are very often
seen as a bi-colored graph, when presented with a non-bipartite graph, the agent should consider
the likelihood of a new category greater than on bipartite graphs.

Additionally, all the presented methods require calculation of $P(x|\overline{novel})$ and
$P(x)$. This way a full model of the graphical model is needed and uncertain sensing as described in
\autoref{sec:cues-from-low-level} cannot be easily incorporated.

This two limitations are a single symptom of the decision on inverting the conditional probability
$P(novel|x)$ to obtain a threshold that can be directly modelled by data.
It is expected that by trying to model how the potentials in the factor graphs change between the
conditional and unconditional graphical model, both limitations can be surpassed.

\section{Future Work}

Future work should try to study and or work around the limitations above presented. At the moment
it is not clear on how the threshold changes when the assumption on constant $P(novel)$ cannot be
made, and attempts to bypass the listed limitations will need to consider that by developing a
method where a static threshold can have a more realistic and controlled behaviour through any graph
structure.

At a longer and larger scale the following paragraphs describe possible and interesting directions
to exploit in the context of detecting knowledge gaps on artificial intelligence systems:


% \subsection*{Extend Novelty to All Variable Types on the Graph}
% Extend novelty to all categories on the graph (room, size, shape, etc..)
% \subsection*{Handle Uncertain and Novel Sensing from Low-level Classifiers}
% Handle uncertain sensing
\subsubsection*{Generalized Framework}
The presented method should be generalized by allowing novelty to be performed on any variable of
graphical model. Additionally the novelty information could be incorporated back in the graph
allowing the agent to probabilistically reason even when variables are considered unknown.
That generalization should aim at being fully probabilistic such as the system presented by
\cite{ranganathan2010pliss} and not deterministic by having to make decisions on which variables
it considers novel.

Additionally several methods exists that allow to produce novelty signals from the low-level
classifiers. A generalized framework should try to fuse and handle all that information by
incorporate it back in the graphical model in a similar fashion on how uncertain sensing is
performed.

% \subsection*{Explain Why a Certain Sample is Novel}
% Explain why a certain sample is novel
% \subsection*{Generation of most Likely Novel Samples}
% Generate most likely novel sample
\subsubsection*{Exploiting Generative Models}
An interesting aspect that arises from the use of generative models is the ability to generate new
samples according to it. It is expected that this can be exploited to achieve a better understanding
of what the system is modelling and what understand common properties between several novel samples
that allow to explain the novelty back to user. For example by generating the most
likely novel samples, it maybe be possible to understand the limitations of its
knowledge and use them for active learning.

% \subsection*{Probabilistic Modelling of Space Segmentation}
% \subsection*{Learn New Concepts}
% Learn Concepts / hidden variable types - (areas, rooms, floors, environments, etc\dots)
\subsubsection*{More than Novel Semantic Categories: Learning Graph Structures}
Though this thesis only touched the problem of detecting novel semantic categories, other knowledge
types should also be considered incomplete and are also candidates for novelty detection and
modelling tools that allow to handle the associated uncertainty on them.
More concretely, besides detecting novel semantic categories on given spatial concepts, there is
also interest on detecting novel concepts: the presence of hidden variables of a new type, not known
for the system: for example detecting room areas or types of environment.

For that an initial step should be made using probabilistic graph structures to model the several
possible space segmentations. Allowing to use a probabilistic approach instead of a deterministic
approach where space is segmented based on the presence of doors or other landmarks.

\subsubsection*{Beyond Detection of Knowledge Gaps}
Having methods to detect gaps of knowledge is just one of the first step on creating long-term and
life-long adaptable systems that are capable of learning through time. After detecting new
situations, ways to learn and incorporating the knowledge back on the agent must be developed.

 

%%----------------------------------------
%% Final materials
%%----------------------------------------

\begin{singlespace}
  %% Bibliography
  %% Comment the next command if BibTeX file not used, 
  %% bibliography is in ``myrefs.bib''
  % \PrintBib{refs}

  % SCons build system does not recognized the bibliography
  % if its nested inside this templates and new commands of
  % feup-teses.
  % For that reason instead of using \PrintBib{Refs}
  % the contents of that command is pasted here.
  \renewcommand{\bibname}{References}%
  \bibliographystyle{alpha}%
  \cleardoublepage%
  \phantomsection%
  \addcontentsline{toc}{chapter}{References}%
  \bibliography{refs}

  %% Index
  %% Uncomment next command if index is required,
  %% don't forget to run ``makeindex mieic'' command
  %\PrintIndex

  %% Comment next 2 commands if numbered appendixes not used
  % \appendix
  % \chapter{Loren Ipsum} \label{ap1:loren}

Depois das conclusões e antes das referências bibliográficas,
apresenta-se neste anexo numerado o texto usado para preencher a
dissertação.

\section{O que é o \emph{Loren Ipsum}?}

\emph{\textbf{Lorem Ipsum}} is simply dummy text of the printing and
typesetting industry. Lorem Ipsum has been the industry's standard
dummy text ever since the 1500s, when an unknown printer took a galley
of type and scrambled it to make a type specimen book. It has survived
not only five centuries, but also the leap into electronic
typesetting, remaining essentially unchanged. It was popularised in
the 1960s with the release of Letraset sheets containing Lorem Ipsum
passages, and more recently with desktop publishing software like
Aldus PageMaker including versions of Lorem Ipsum~\citep{kn:Lip08}. 

\section{De onde Vem o Loren?}

Contrary to popular belief, Lorem Ipsum is not simply random text. It
has roots in a piece of classical Latin literature from 45 BC, making
it over 2000 years old. Richard McClintock, a Latin professor at
Hampden-Sydney College in Virginia, looked up one of the more obscure
Latin words, consectetur, from a Lorem Ipsum passage, and going
through the cites of the word in classical literature, discovered the
undoubtable source. Lorem Ipsum comes from sections 1.10.32 and
1.10.33 of ``de Finibus Bonorum et Malorum'' (The Extremes of Good and
Evil) by Cicero, written in 45 BC. This book is a treatise on the
theory of ethics, very popular during the Renaissance. The first line
of Lorem Ipsum, ``Lorem ipsum dolor sit amet\ldots'', comes from a line in
section 1.10.32.

The standard chunk of Lorem Ipsum used since the 1500s is reproduced
below for those interested. Sections 1.10.32 and 1.10.33 from ``de
Finibus Bonorum et Malorum'' by Cicero are also reproduced in their
exact original form, accompanied by English versions from the 1914
translation by H. Rackham.

\section{Porque se usa o Loren?}

It is a long established fact that a reader will be distracted by the
readable content of a page when looking at its layout. The point of
using Lorem Ipsum is that it has a more-or-less normal distribution of
letters, as opposed to using ``Content here, content here'', making it
look like readable English. Many desktop publishing packages and web
page editors now use Lorem Ipsum as their default model text, and a
search for ``lorem ipsum'' will uncover many web sites still in their
infancy. Various versions have evolved over the years, sometimes by
accident, sometimes on purpose (injected humour and the like). 

\section{Onde se Podem Encontrar Exemplos?}

There are many variations of passages of Lorem Ipsum available, but
the majority have suffered alteration in some form, by injected
humour, or randomised words which don't look even slightly
believable. If you are going to use a passage of Lorem Ipsum, you need
to be sure there isn't anything embarrassing hidden in the middle of
text. All the Lorem Ipsum generators on the Internet tend to repeat
predefined chunks as necessary, making this the first true generator
on the Internet. It uses a dictionary of over 200 Latin words,
combined with a handful of model sentence structures, to generate
Lorem Ipsum which looks reasonable. The generated Lorem Ipsum is
therefore always free from repetition, injected humour, or
non-characteristic words etc. 

\end{singlespace}

\end{document}
