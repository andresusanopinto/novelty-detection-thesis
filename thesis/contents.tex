% What should the title be?
% Novelty Detection for Visual Indoor Categorization
% Novelty Detection on Semantic Representations.


\chapter{Introduction}
% Introduce:
%  Mobile Robotics, A.I.
%  Interaction with Humans
%  Need for concepts/semantic information
%  Dynamic human environment
%  Non-feasibility of hard-coded concepts
%  Requirements to detect novelty and handle it.
For a long time humanity has fantasized that one day robots will walk among us.
They will move and be able to interact with us. Understand our concepts and
be able to reason.

They need to possess ability to adapt to situations as its infeasible to rely
on extensive man-work to tag objects, map space and code all the aspects that
make up our human-reality.

% == Introduce the need of semantic representations ==
%
% There is a lot of low-level methods (or specific knowledge to robots)
% But the creation of semantic representations allow to bridge that low-level
% sensing with high-level concepts that facilitate several high-level Tasks.


% Importance of mapping and localization in a robot as well as reasoning on
% properties of each space.
% Humans give labels to space that characterized the properties and activities
% expected to be performed on them.
% Such information is of great interest to be structured and organized in such
% a way reasoning and planning can be implemented.
%
% It also holds an important role in long-term planning and stability as it
% hides the space-time local details and allows to focus on high-level thinking.

% Deal with uncertainty -> need for probabilistic models.
% Deal with semantic    -> need of structured models such as graphical models.


Example of what kind of an high-level thinking on space and its semantics can
provide: Get me a beer/milk concept.

% == Reasoning about need for Novelty Detection ==
%
% Dynamic environment (inability to know the environment or even to be possible
% to map all semantic concepts to interact with humans).
%
% Knowledge-Awareness (knowing what we know... And detecting novel cases)
% Identify gaps in the semantic knowledge.
% Automatic detection and learning of novel concepts.
%
% Increase robustness
% Self-Extending
%


\section{Problem and Goals}
% Extend a probabilistic graphical modelling framework with capabilities to
% detect novelty.
% Both in terms of novel classes as of novel structure.
%
Given a probabilistic structure representing the sensed conceptual knowledge
obtained by the agent develop methods to be able to detect novelty present
either on new semantic concepts (new classes) or even on structure that was
previously unknown.


% Yeah thats what you call dream :P
In an extreme ideal case the agent should be able to go from zero-knowledge to
understanding presence of certain objects (as generators of a set of sensed
sensed properties grouped locally), understand areas within rooms (sink-area on
on a kitchen), rooms as separated by doors, and understand environments such
as office, home, warehouse, spaceship.


For that the probabilistic semantic representation presented by
\cite{andrzej2011phd} is used.

% If everything goes fine... 2 of the "Future Directions" proposed are 
% Novelty Detection and Learning of Novel Concepts:
%  Identify Gaps in Spatial and Semantic Knowledge -> Addressed
%  Performing learning of new concepts -> Outside Scope / No Time
%
% Using Properties for Space Segmentation
%  If we move on detecting novel structures on the graph we are addressing this
%  issue.



\section{Thesis Outline}
The rest of this thesis is structured as follows:

\autoref{chap:background} introduces background concepts for understanding the
presented thesis. In particular \autoref{sec:graphical-models} introduces
\emph{probabilistic graphical models} that lay the base modelling tool for
the semantic structured representation over which novelty detection
methods are developed.

\autoref{chap:semantic-mapping} describes the system proposed by
\cite{andrzej}. In special it introduces the \emph{conceptual map} of the system
that is responsible for the creation of the \emph{probabilistic semantic
representation} that this thesis aims at improving by developing methods with
the capability to identify knowledge gaps.

\autoref{chap:novelty-intro} introduces novelty detection under a
statistical view point and how to interpret it as a threshold function.
A simple approach on how to perform novelty on very simple graphs
is given together with some results on the impact of using semi-supervised
novelty detection to improve the system performance.

\autoref{chap:novelty} presents the developed techniques developed to identify
novelty on the semantic representation.

\autoref{chap:conclusions} draws conclusions on the developed work and presents
interesting directions for future works.



%%%%%%%%%%%%%%%%%%%%%%%%%%%%%%%%%%%%%%%%%%%%%%%%%%%%%%%%%%%%%%%%%%%%%%%%%%
%%%%%%%%%%%%%%%%%%%%%%%%%%%%%%%%%%%%%%%%%%%%%%%%%%%%%%%%%%%%%%%%%%%%%%%%%%
\chapter{Background}\label{chap:background}
% This chapter will perform a breath-first explanation on the contents related
% with the thesis. The extension of the contents explained here can be seen as
% monotone function over the number of missing pages to attain the requirements.
%
% Some topics are indeed important as its the case of factor graphs.
% Others like:
%  - classification (SVM, multi-class, kernel-trick),
%  - features (SIFT, GIST),
%  - Basic Bayes theory and probabilistic review
% are included for fun of the reader in case it inquires himself how to deal
% with the low-levels features we propose using.
%
% ---
% PS: This is the chapter that will be used to fill with unnecessary crap
% in case they really annoy me with the number of pages. Every other chapter
% will strive to be really required and be a masterpiece.
%

\section{Classification}
\subsection{Recognition and Categorization}
\subsection{Support Vector Machines}
\subsection{Multi-Class Classification}
%\subsection{Kernel-Trick}

\section{Visual Features}
\subsection{Local Features}
\subsection{Global Features}

%%%%%%%%%%%%%%%%%%%%%%%%%%%%%%%%%%%%%%%%%%%%%%%%%%
% Important sections of this chapter begin here.
%%%%%%%%%%%%%%%%%%%%%%%%%%%%%%%%%%%%%%%%%%%%%%%%%%
\section{Probability Theory}


\section{Principle of Maximum Entropy}
The principle of maximum entropy states that given a set of distributions that
are coherent with the acquired knowledge, the one which maximizes entropy should
be picked.

In the case of non-available information this is the uniform distributions.
In cases where we only know the mean and standard deviation a normal
distribution shall be picked, etc\dots

\section{Probabilistic Graphical Models}
\label{sec:graphical-models}
\subsection{Factor Graphs}
\subsection{Inference Engines}
Exact Inference and Approximate Inference methods

% In case we expand in some finding structure in graphs maybe the following will
% be useful
\section{Label Propagation}
\section{Min-Cuts}


%%%%%%%%%%%%%%%%%%%%%%%%%%%%%%%%%%%%%%%%%%%%%%%%%%%%%%%%%%%%%%%%%%%%%%%%%%
%%%%%%%%%%%%%%%%%%%%%%%%%%%%%%%%%%%%%%%%%%%%%%%%%%%%%%%%%%%%%%%%%%%%%%%%%%
\chapter{Semantic Mapping}\label{chap:semantic-mapping}

% Advantages of Multi-modal approaches
% 

\section{Dora Architecture Overview}
\subsection{System organization}
\subsubsection{Sensory Layer}
\subsubsection{Categorical Layer}
\subsubsection{Place Layer}
\subsubsection{Conceptual Layer}

\section{Features}
\section{Conceptual Knowledge}
\section{Conceptual Map}

%%%%%%%%%%%%%%%%%%%%%%%%%%%%%%%%%%%%%%%%%%%%%%%%%%%%%%%%%%%%%%%%%%%%%%%%%%
%%%%%%%%%%%%%%%%%%%%%%%%%%%%%%%%%%%%%%%%%%%%%%%%%%%%%%%%%%%%%%%%%%%%%%%%%%
\chapter{Novelty Detection}\label{chap:novelty-intro}
\section{Novelty Detection as a Threshold}
\subsection{Conditional Probability}
\subsection{Unconditional Probability}

\section{Semi-Supervised Novelty Detection: Using Unlabelled Data}

%%%%%%%%%%%%%%%%%%%%%%%%%%%%%%%%%%%%%%%%%%%%%%%%%%%%%%%%%%%%%%%%%%%%%%%%%%
%%%%%%%%%%%%%%%%%%%%%%%%%%%%%%%%%%%%%%%%%%%%%%%%%%%%%%%%%%%%%%%%%%%%%%%%%%
\chapter{Novelty Detection on Semantic Representations}\label{chap:novelty}
\section{Detecting New Classes} 
\section{Detecting New Structure}

%%%%%%%%%%%%%%%%%%%%%%%%%%%%%%%%%%%%%%%%%%%%%%%%%%%%%%%%%%%%%%%%%%%%%%%%%%
%%%%%%%%%%%%%%%%%%%%%%%%%%%%%%%%%%%%%%%%%%%%%%%%%%%%%%%%%%%%%%%%%%%%%%%%%%
\chapter{Conclusions and Future Work}\label{chap:conclusions}
\section{Future Work}

