\chapter*{Abstract}

For a long time humanity has dreamed that one day robots will be among us. They will explore our world
and interact with us, understand our concepts and reason. An important step in that direction is endowing 
robots with knowledge about human concepts and semantics. However, it is unrealistic to believe that the 
human world can be fully modeled in the robot's brain at the design stage. Therefore, robots must be able 
to adapt and learn when confronted with novel situations. Detection of novel situations, where the knowledge
of the robot is not sufficient plays an important role in adaptation and learning of new concepts and 
is the main topic of this thesis.

In the context of mobile robotics, spatial concepts and semantics are crucial to enable the robot to perform
complex human-like tasks and human interactions. For handling those, a robot builds a representation
of space extended with semantic properties, process which is known as semantic mapping.
This representation identifies spatial entities and classifies them according to their meaning to
humans allowing the robot to reason at a high abstraction level. For example, humans categorize spaces as 
kitchens, bedrooms, corridors, offices, theaters, computer labs, etc. based on spatial properties, 
objects and actions that are characteristic of those spaces. Similarly, if a robot had a knowledge about
those spatial concepts, it could answer such questions as: ``Where are cornflakes?''

This thesis studies a semantic mapping process that employs spatial semantic knowledge
and represents it using a probabilistic graphical model. It develops methods 
for performing detection of novel room categories which were not previously known to the robot,
using for that graphical models. Finally, it draws attention to the usefulness of unlabelled data for the 
novelty detection process in order to considerably increase the detection accuracy.

% Portuguese Version of the above.

\chapter*{Resumo}
\selectlanguage{portuguese}

Já há muitos anos que a humanidade sonha que um dia robôs caminharam lado a lado com os humanos.
Acredita-se que eles serão capaz de se mover e interagir connosco, compreender os nossos conceitos
e serem capazes de pensar.
Um passo importante nesta direcção é a criação de robôs com conhecimento de conceitos e significados
próprios do mundo humano.
No entanto é irrealista acreditar que será possível descrever e representar em toda a sua extensão
tais conceitos. Assim os agentes deverão ser capazes de aprender e adaptar o seu conhecimento quando
confrontados com novas situações.
A detecção destas novas situações, onde o conhecimento de um agente não é suficiente, têm um papel importante
na adaptação e aprendizagem de novos conceitos e significados e é por isso o tópico principal desta tese.

No contexto de agentes moveis, para facilitar o deslocamento e planeamento, os
agentes constroem uma representação do ambiente. Quando essa representação é estendida com
conceitos e categorias com significado para humanos, as tarefas de alto nível ficam mais fáceis.
Por exemplo, a uma escala de quartos, os humanos categorizam o espaço em: cozinhas, quartos, corredores,
escritórios, etc\dots
Ao relacionar quartos com estas categorias que descrevem propriedades e acções percebidas por
humanos, um agente consegue resolver tarefas como: ``Onde estão os cereais?''

Esta tese estuda um processo, que usa conhecimento de conceitos espaciais e categorias com significado
para humanos, para realizar mapeamento semântico do ambiente através de modelos gráficos probabilísticos.
Apresenta depois técnicas para realizar detecção de novas categorias. Chama também a atenção para a
utilização de dados não supervisionados para efectivamente melhorar a exactidão da detecção.

\selectlanguage{english}

