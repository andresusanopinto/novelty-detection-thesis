\chapter*{Abstract}
For a long time humanity has fantasized that one day robots will walk among us.
They will move and be able to interact with us, understand our concepts and
be able to reason.
An important step on that direction is endowing robots with conceptual knowledge
of human semantics. However its unrealistic to believe on the possibility of extensively
describing human concepts to robots on a pre-deploy phase, and so, robots must be able to adapt
and learn when confronted with novel situations.
Detection of novel situations, where the robot knowledge is not sufficient, plays an important
role for the subsequent adapting and learning of new concepts and its this thesis main topic.

In the context of mobile robotics, spatial concepts and semantics play a main role to perform
complex human-like tasks and human interactions, for handling that a robot builds a representation
of space extended with semantic properties, process which is known as semantic mapping.
More exactly the ability to relate areas with human semantics describing expected properties
and actions that take place on it allow the robot to reason at a very high-level close to humans.
For example in a room scale level, humans categorize the space in kitchens, bedrooms, corridors,
office, theaters, etc\dots

This thesis studies a proposed %~\cite{pronobis2011semmap}
process that utilizes spatial and semantic knowledge for semantic mapping using a
probabilistic graphical model,
and develops methods to perform detection of room categories not previously
known to the robot using that graphical model.
At last it draws attention to the use of unlabelled data on the novelty detection process,
which can be used to considerably increase performance.


% Portuguese Version of the above.
\chapter*{Resumo}
Assume this text to be $T_{E \rightarrow P}(abstract)$.
\footnote{What I mean is that this will be a translation of the English text
present on abstract. Also don't expect it to be filled in before the final
version of this document!}

