\chapter*{Abstract}

For a long time humanity has dreamed that one day robots will be among us. They will explore our world
and interact with us, understand our concepts and reason. An important step in that direction is endowing 
robots with knowledge about human concepts and semantics. However, it is unrealistic to believe that the 
human world can be fully modeled in the robot's brain at the design stage. Therefore, robots must be able 
to adapt and learn when confronted with novel situations. Detection of novel situations, where the knowledge
of the robot is not sufficient plays an important role in adaptation and learning of new concepts and 
is the main topic of this thesis.

In the context of mobile robotics, spatial concepts and semantics are crucial to enable the robot to perform
complex human-like tasks and human interactions. For handling those, a robot builds a representation
of space extended with semantic properties, process which is known as semantic mapping.
This representation identifies spatial entities and classifies them according to their meaning to
humans allowing the robot to reason at a high abstraction level. For example, humans categorize spaces as 
kitchens, bedrooms, corridors, offices, theaters, computer labs, etc. based on spatial properties, 
objects and actions that are characteristic of those spaces. Similarly, if a robot had a knowledge about
those spatial concepts, it could answer such questions as: ``Where are cornflakes?''

This thesis studies a semantic mapping process that employs spatial semantic knowledge
and represents it using a probabilistic graphical model. It develops methods 
for performing detection of novel room categories which were not previously known to the robot applicable 
to that graphical model. Finally, it draws attention to the usefullness of unlabelled data for the 
novelty detection process in order to considerably increase the detection accuracy.

% Portuguese Version of the above.
\chapter*{Resumo}
Assume this text to be $T_{E \rightarrow P}(abstract)$.
\footnote{What I mean is that this will be a translation of the English text
present on abstract. Also don't expect it to be filled in before the final
version of this document!}

