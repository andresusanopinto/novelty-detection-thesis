%-----------------------------------------------
% Template para criação de resumos de projectos/dissertação
% jlopes AT fe.up.pt,   Fri Jul  3 11:08:59 2009
%-----------------------------------------------

\documentclass[9pt,a4paper]{extarticle}

%% English version: comment first, uncomment second
%\usepackage[portuguese]{babel}  % Portuguese
\usepackage[english]{babel}     % English
\usepackage{graphicx}           % images .png or .pdf w/ pdflatex OR .eps w/ latex
\usepackage{times}              % use Times type-1 fonts
\usepackage[utf8]{inputenc}     % 8 bits using UTF-8
\usepackage{url}                % URLs
\usepackage{multicol}           % twocolumn, etc
\usepackage{float}              % improve figures & tables floating
\usepackage[tableposition=top]{caption} % captions
%% English version: comment first (maybe)
%\usepackage{indentfirst}        % portuguese standard for paragraphs
%\usepackage{parskip}

%% page layout
\usepackage[a4paper,margin=30mm,noheadfoot]{geometry}

%% space between columns
\columnsep 12mm

%% headers & footers
\pagestyle{empty}

%% figure & table caption
\captionsetup{figurename=Fig.,tablename=Tab.,labelsep=endash,font=bf,skip=.5\baselineskip}

%% heading
\makeatletter
\renewcommand*{\@seccntformat}[1]{%
  \csname the#1\endcsname.\quad
}
\makeatother

%% avoid widows and orphans
\clubpenalty=300
\widowpenalty=300

% Additional packages
\usepackage[utf8]{inputenc}
\usepackage{subfig}
\usepackage[pdftex,pdfpagelabels,bookmarks,hyperindex,hyperfigures]{hyperref}
\hypersetup{%
   plainpages=false, 
   pdfpagelayout=SinglePage,
   bookmarksopen=false,
   bookmarksnumbered=true,
   breaklinks=true,
   linktocpage,
   colorlinks=true,
   linkcolor=blue,
   urlcolor=blue,
   citecolor=blue,
   anchorcolor=green
}      



\begin{document}

\title{\vspace*{-8mm}\textbf{\textsc{Novelty Detection for Semantic Place Categorization}}}
\author{\emph{André Susano Pinto}$^\dag$\\[2mm]
        \small{Supervised by: Andrzej Pronobis$^\ddag$, Luis Paulo Reis$^\dag$}\\
        \small{$^\dag$Faculdade de Engenharia da Universidade do Porto, Portugal}\\
        \small{$^\ddag$Royal Institute of Technology (KTH), Stockholm, Sweden}}
%\url{andresusanopinto@gmail.com}\\
\date{}
\maketitle
%no page number 
\thispagestyle{empty}

\vspace*{-4mm}\noindent\rule{\textwidth}{0.4pt}\vspace*{4mm}

\begin{multicols}{2}

\section{Motivation}
For a long time humanity has fantasized that one day robots will walk among us.
They will move and be able to interact with us, understand our concepts and
be able to reason.
An important step on that direction is endowing robots with knowledge of human concepts and
semantics. However its unrealistic to believe on the possibility of extensively
describing human concepts and semantics to robots on a pre-deploy phase, and so, robots must be able
to adapt and learn when confronted with novel situations.
Detection of novel situations, where the robot knowledge is not sufficient, plays then an important
role for the subsequent adapting and learning of new concepts and its this thesis main topic.

Additionally in the context of mobile robotics, spatial concepts and semantics play a main role to perform
complex human-like tasks and human interactions. For handling those, a robot builds a representation
of space extended with semantic properties, process which is known as semantic mapping.
This representation identifies spatial entities and classifies them according to their meaning to
humans allowing the robot to reason at a very high-level.
For example, in a room scale, humans categorize space in kitchens, bedrooms, corridors,
office, theater, computer lab, etc\dots, by relating room areas with those categories, that describing expected
properties and actions that take place on them, a robot can solve tasks like:
``Where are the cornflakes?''

\section{Problem and Goals}
This thesis focus then on the particular problem of identifying novel semantic categories for places
using
This thesis studies a proposed %~\cite{pronobis2011semmap}
process that utilizes spatial and semantic knowledge for semantic mapping using a
probabilistic graphical model,
and develops methods to perform detection of room categories not previously
known to the robot using that graphical model.

\section{Semantic Mapping}
Semantic mapping is the process by which an agent builds a representation of the environment and
enhances the identified entities with semantic values (i.e.\ human semantics).
In this thesis the semantic mapping process proposed by \cite{pronobis2011phd} was used as based.

\subsection{Spatial Knowledge Representation}
In order to perform semantic mapping an agent needs to possess knowledge on how to segment, identify
entities existent in the space by analysing the sensed properties. For that Pronobis~\cite{pronobis2011semmap} presents a
layered knowledge structure:
the sensory layer maintains an accurate and exact description of environment in a machine-friendly representation,
the place layer employs knowledge on how to segment and identify areas such as rooms,
the categorical layer defines knowledge on properties and categories that allow to extract
higher-level information from the sensed data
and the conceptual layer maintains knowledge on entities and their relations between them as well
information on how to map those to human semantics.

\subsection{Conceptual Map}
The semantic mapping is then based on the instantiation of a conceptual map that uses all the
ontology defined on the conceptual layer in order to build a probabilistic graphical model that
relates and propagates probabilities across all the entities detect and properties detected by the
other layers.

By using probabilistic graphical models, the conceptual map becomes then a generative model that
allows several inferences on variable configurations to be performed and for that serves as a useful
tool for novelty detection of unknown categories of some of its variables.

\section{Novelty Detection}
Novelty detection, also known as outlier or anomaly detection, is a
classification problem related to identification of new or unknown data
patterns that the system is not aware of~\cite{markou2003novelty}.
The ability to identify novel cases is crucial in any autonomous system
that is deployed to unknown or to uncontrolled environments, as it gives the
system the ability to detect that something is not conforming to its knowledge and
therefore should be treated with caution.
% It has several applications such as fault detection~\cite{tarassenko1999novelty},
% intrusion detection~\cite{fan2001using},
% detection of masses in mammograms~\cite{tarassenko1995novelty} or detection of
% novel and useful documents~\cite{zhang2002novelty}.

\subsection{Novelty Detection as Threshold}
Due to the noisy, uncertain and insufficient nature of sensed data, novelty ought to be
treated in a probabilistic way in where each sample has a certain probability
of being generated by a class not known for the agent and a complementary probability $P(\overline{novel}|x)$
of being generated by a known class.

Additionally any detector can be uniquely described by the set $N$ of samples that it
classifies as novel. With that its possible to show that by including a new sample $a$ in $N$ an agent
can increase its detection rate at a cost of producing more mis-classifications.
This describes the base of the \emph{error and rejection tradeoff} described by \cite{chow1970optimum}.
Due to the interest of having the highest detection rate with a given minimum error rate, any optimal
detector can be described as a \emph{continuous knapsack problem} of the sample space.
The greedy ordering defined to solve it can be shown to be equivalent to:
\begin{eqnarray}
P(\overline{novel}|x) &=& \frac{P(x|\overline{novel})P(\overline{novel})}{P(x)}
\end{eqnarray}

\subsection{Assumption on a constant $P(novel)$}
The ratio introduced on the previous section allows to implement an optimal detector.
Though the factors involved need to be defined according to available data.

$P(\overline{novel})$ was considered constant through all possible situations. Although this
sounds a weak assumption note that in the case hat uses the ratio

\subsection{Modeling Conditional Probability}
Since the graphical model produced by the conceptual map tries to model the distribution of
variables assuming the knowledge of the agent holds true, it represents a method to model
$P(x|\overline{novel})$ since it allows to calculate the density probability that the set of features $x$ is
sensed given that all the variables and graph structure, including the category of
a specific room $a$ are correctly modelled by the agent knowledge.
This way the distribution modelled by a factor graph equivalent to the chain-graph used by
the conceptual map can be used as an approximation for the conditional probability of $x$.

\subsection{Modeling Unconditional Probability}
With no knowledge on the real distribution of all samples, the correct approach is to model it
with a uniform distribution. Nonetheless in the presented case, it was assumed that only the room
category of a specific room is unknown and that the remaining knowledge holds trues.
In that case, the correct approach is to replace the variable $a$, that represents the room where to
detect novelty, with a single factor connecting all variables directly dependent on it. 

Due to its high-dimensionality, approximating that factor with unlabelled data is infeasible and
approximations have to be done. In this thesis work two approximations where tried: an uniform model
and an independent model as illustrated in \autoref{fig:models}. In the independent model, the
single\hyp{}connected factors are trained from unlabelled data and its expected that they create a better
approximation to the unconditional probability than an assumption on the uniform model.

\section{Results}
A synthetic dataset modelling the expected features when implementing on the real case and enforcing
a constant $P(novel)$ was used for testing the developed models.
The two suggested models where compared between them and to an optimal case across several situations
showing they behaviour as more information is sensed from the environment.

\begin{figure}[H]
\centering
\includegraphics[width=0.4\textwidth]{results/synthetic-all.pdf}
\caption{\label{fig:results}Comparison of the proposed models using samples from a synthetic dataset
         with variable number of sensed features.}
\end{figure}

\section{Conclusions}
This thesis studied the problem of novelty detection of semantic categories on the spatial knowledge
of mobile robots. To that purpose a semantic mapping process~\cite{pronobis2011semmap} was studied and
a novelty detection scheme based on the conceptual map used by it proposed.
Initial tests where performed on synthetic data showing an interest in using unlabelled data to
increase the detection performance.
Additionally all the assumptions and bases behind novelty detection were presented and several
directions on how to create a generalized framework for novelty detection were discussed.


\bibliographystyle{unsrt}
\bibliography{refs}

\end{multicols}

\end{document}
