%-----------------------------------------------
% Template para criação de resumos de projectos/dissertação
% jlopes AT fe.up.pt,   Fri Jul  3 11:08:59 2009
%-----------------------------------------------

\documentclass[9pt,a4paper]{extarticle}

%% English version: comment first, uncomment second
%\usepackage[portuguese]{babel}  % Portuguese
\usepackage[english]{babel}     % English
\usepackage{graphicx}           % images .png or .pdf w/ pdflatex OR .eps w/ latex
\usepackage{times}              % use Times type-1 fonts
\usepackage[utf8]{inputenc}     % 8 bits using UTF-8
\usepackage{url}                % URLs
\usepackage{multicol}           % twocolumn, etc
\usepackage{float}              % improve figures & tables floating
\usepackage[tableposition=top]{caption} % captions
%% English version: comment first (maybe)
%\usepackage{indentfirst}        % portuguese standard for paragraphs
%\usepackage{parskip}

%% page layout
\usepackage[a4paper,margin=30mm,noheadfoot]{geometry}

%% space between columns
\columnsep 12mm

%% headers & footers
\pagestyle{empty}

%% figure & table caption
\captionsetup{figurename=Fig.,tablename=Tab.,labelsep=endash,font=bf,skip=.5\baselineskip}

%% heading
\makeatletter
\renewcommand*{\@seccntformat}[1]{%
  \csname the#1\endcsname.\quad
}
\makeatother

%% avoid widows and orphans
\clubpenalty=300
\widowpenalty=300

% Additional packages
\usepackage[utf8]{inputenc}
\usepackage{subfig}
\usepackage[pdftex,pdfpagelabels,bookmarks,hyperindex,hyperfigures]{hyperref}
\hypersetup{%
   plainpages=false, 
   pdfpagelayout=SinglePage,
   bookmarksopen=false,
   bookmarksnumbered=true,
   breaklinks=true,
   linktocpage,
   colorlinks=true,
   linkcolor=blue,
   urlcolor=blue,
   citecolor=blue,
   anchorcolor=green
}      



\begin{document}

\title{\vspace*{-8mm}\textbf{\textsc{Novelty Detection for Semantic Place Categorization}}}
\author{\emph{André Susano Pinto}$^\dag$\\[2mm]
        \small{Supervised by: Andrzej Pronobis$^\ddag$, Luis Paulo Reis$^\dag$}\\
        \small{$^\dag$Faculdade de Engenharia da Universidade do Porto, Portugal}\\
        \small{$^\ddag$Royal Institute of Technology (KTH), Stockholm, Sweden}}
%\url{andresusanopinto@gmail.com}\\
\date{}
\maketitle
%no page number 
\thispagestyle{empty}

\vspace*{-4mm}\noindent\rule{\textwidth}{0.4pt}\vspace*{4mm}

\begin{multicols}{2}

\section{Motivation}
For a long time humanity has dreamed that one day robots will be among us. They will explore our world
and interact with us, understand our concepts and reason. An important step in that direction is endowing 
robots with knowledge about human concepts and semantics. However, it is unrealistic to believe that the 
human world can be fully modeled in the robot's brain at the design stage. Therefore, robots must be able 
to adapt and learn when confronted with novel situations. Detection of novel situations, where the knowledge
of the robot is not sufficient plays an important role in adaptation and learning of new concepts and 
is the main topic of this thesis.

In the context of mobile robotics, spatial concepts and semantics are crucial to enable the robot to perform
complex human-like tasks and human interactions. For handling those, a robot builds a representation
of space extended with semantic properties, process which is known as semantic mapping.
This representation identifies spatial entities and classifies them according to their meaning to
humans allowing the robot to reason at a high abstraction level. For example, humans categorize spaces as 
kitchens, bedrooms, corridors, offices, theaters, computer labs, etc. based on spatial properties, 
objects and actions that are characteristic of those spaces. Similarly, if a robot had a knowledge about
those spatial concepts, it could answer such questions as: ``Where are cornflakes?''

\section{Problem and Goals}
The semantic mapping process plays an important role for allowing artificial mobile agents to reason at 
a high level and is dependent on the knowledge that the agent has about the spatial concepts and semantic 
categories. For that reason there is high interest in creating methods that are capable of detecting 
gaps in the agent's knowledge. For that purpose, this thesis, studies novelty detection techniques and 
how they can be applied to detection novel room categories during the semantic mapping process.
To this end, the semantic mapping process proposed in \cite{pronobis2011phd} is studied and a method
to detect novel categories using graphical models is suggested.

\section{Semantic Mapping}
Semantic mapping is the process by which an agent builds a representation of the environment and
augments the identified spatial entities with semantic values (i.e.\ human semantics).
In this thesis the semantic mapping process proposed in \cite{pronobis2011phd} is used as base.

\subsection{Spatial Knowledge Representation}
In order to perform semantic mapping an agent needs to possess knowledge on how to segment and
identify entities existent in space by analysing the sensed data. For that, 
Pronobis~\cite{pronobis2011semmap} presents a spatial knowledge representation consisting of four layers:
the \emph{sensory layer} maintains an accurate and exact description of the environment in a machine-friendly 
representation, the \emph{place layer} represents the world in terms of a set of discrete places and connectivity
between them, the \emph{categorical layer} defines knowledge about properties and categories that allow to extract
higher-level information from the sensed data, and the \emph{conceptual layer} maintains knowledge about 
spatial entities and concepts as well as the relations between them.

\subsection{Conceptual Map}
The semantic mapping is based on the instantiation of a conceptual map that uses the
ontology defined in the conceptual layer in order to build a probabilistic graphical model that
relates and propagates probabilities across all the entities and properties detected by the
other layers.

By using probabilistic graphical models, the conceptual map becomes a generative model that
permits inference about different variables. This enables the use of the conceptual map 
as a tool for detection of unknown categories.

\section{Novelty Detection}
Novelty detection, also known as outlier or anomaly detection, is a
classification problem related to identification of new or unknown data
patterns that the system is not aware of~\cite{markou2003novelty}.
The ability to identify novel cases is crucial in any autonomous system
that is deployed in unknown or uncontrolled environments, as it permits 
detection of cases that do not conform to the robot's knowledge and
therefore should be treated with caution.
% It has several applications such as fault detection~\cite{tarassenko1999novelty},
% intrusion detection~\cite{fan2001using},
% detection of masses in mammograms~\cite{tarassenko1995novelty} or detection of
% novel and useful documents~\cite{zhang2002novelty}.

\subsection{Novelty Detection as Thresholding}
Due to the nature of the sensed data which are noisy and uncertain, novelty ought to be
treated in a probabilistic way where each sample has certain probability
of being generated by a class not known to the agent and a complementary probability $P(\overline{novel}|x)$
of being generated by a known class.

Additionally, any detector can be uniquely described by the set $N$ of samples that it
classifies as novel. With that, it is possible to show that by including a new sample $a$ into 
$N$, an agent can increase its detection rate at a cost of producing more misclassifications.
This describes the base of the \emph{error and rejection tradeoff} described in \cite{chow1970optimum}.
Due to the goal of having the highest detection rate with a given minimum error rate, any optimal
detector can be described as a \emph{continuous knapsack problem} of the sample space.
The resulting greedy ordering can be shown to be equivalent to:
\begin{eqnarray}
P(\overline{novel}|x) &=& \frac{P(x|\overline{novel})P(\overline{novel})}{P(x)}
\end{eqnarray}

\subsection{Assumption about Constant $P(novel)$}
By considering $P(novel)$ to be constant, a ratio between conditional and unconditional probabilities
of the sensed variables is obtained. Although it sounds reasonable to assume that the probability of a specific 
room belonging to a novel category is constant, this thesis points out that this assumption might be very strong 
and not reasonable in realistic scenarios. The main problem is related to the fact that the graph structure
changes as the robot gathers new samples and some structures might be more prone to produce novel samples. This
is different from the normal classification scenarios where its clear that the category is indeed the only 
generator of features.

\subsection{Modeling Conditional Probability}
Since the graphical model produced by the conceptual map tries to model the distribution of
variables assuming that the knowledge of the agent is representative of the world, it models $P(x|\overline{novel})$.
%since it allows to calculate the density probability that the set of features $x$ is
%sensed given that all the variables and graph structure, including the category of
%a specific room $a$ are correctly modelled by the agent knowledge.
%This way the distribution modelled by a factor graph equivalent to the chain-graph used by
%the conceptual map can be used as an approximation for the conditional probability of $x$.

\subsection{Modeling Unconditional Probability}
With no knowledge about the real distribution of samples, the correct approach is to model it
with a uniform distribution. Nonetheless in the presented case, it was assumed that only the room
category of a specific room is unknown and that the remaining knowledge holds true.
In that case, the correct approach is to replace the variable that represents the room where the 
novelty is detected with a single factor connecting all variables directly dependent on it. 

Due to its high-dimensionality, approximating that factor with unlabelled data is infeasible and
approximations have to be done. In this thesis two approximations were presented: a uniform model
and an independent model. In the independent model, the single\hyp{}connected factors are trained from 
unlabelled data and its expected that they create a better approximation to the unconditional probability 
than an assumption about the uniform model.

\section{Results}
A synthetic dataset modelling the features expected when deployed in a real scenario and enforcing
a constant $P(novel)$ was used for testing the developed models. The two suggested models where compared 
to each other and to an optimal case. The tests across several situations show how the behaviour changes 
as more information is sensed from the environment.

\begin{figure}[H]
\centering
\includegraphics[width=0.4\textwidth]{results/synthetic-all.pdf}
\caption{\label{fig:results}Comparison of the proposed models using samples from a synthetic dataset
         with variable number of sensed features.}
\end{figure}

\section{Conclusions}
This thesis studied the problem of detection of novel semantic categories of rooms in the context of
semantic mapping with mobile robots. To that purpose, a semantic mapping process~\cite{pronobis2011semmap} 
was studied and a novelty detection scheme based on a conceptual map was proposed.
Initial tests where performed on synthetic data showing benefit of using unlabelled data to
increase the detection performance. Additionally, all the assumptions behind novelty detection were 
presented and several directions on how to create a generalized framework for novelty detection 
were discussed.

\bibliographystyle{unsrt}
\bibliography{refs}

\end{multicols}

\end{document}
