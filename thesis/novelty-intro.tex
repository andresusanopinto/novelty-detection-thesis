\chapter{Novelty Detection}\label{chap:novelty-intro}

This chapter presents concepts that are used as base when adding novelty
detection capabilities on the semantic representation in \autoref{chap:novelty}.
It starts by providing the reader with a brief overview on novelty detection
techniques and related works.

After it shows that an optimal novelty detection system can be implemented by
thresholding on a order-relation defined over the inputs.
And that an equivalent order-relation can be imposed by a ratio between
conditional and unconditional probabilities.
Some discussion on how to interpret the meaning of those factors is also
presented.

At last a practical example using semantic data and probabilistic graphical
models is presented: it shows how to use the introduced ratio to obtain a
novelty detection system and analyses the performance impact by increasing
the amount of sensed data and by using an approximation on the unconditional
probability.


\section{Novelty Detection Review and Related Work}
Novelty detection, also called outlier or anomaly detection, is the
classification problem related with identification of new or unknown data
patterns that the system is not aware of~\cite{markou2003novelty}.

The ability to identify novel cases is crucial in any autonomous system
that is deployed to unknown or to uncontrolled environments, as it gives the
system the ability to detect something is not conforming to it knowledge and
therefore should proceed with caution.
It has several applications such as fault detection, intrusion detection,
detection of masses in mammograms, hand written digit recognition and many
others.
% TODO: list usage of novelty detection together with references to papers
% with them.

It is on the nature of unknown environments and/or in the infeasibility of
training the system on all possible object classes that lies the complexity of
novelty detection: it is only possible to obtain samples representing positives
examples of known cases and the lack of negative examples renders normal
classification methods unusable.

In order to be usable in real world applications, novelty detection methods
have to overcome a series of obstacles:
be able to generalize while still detecting novelty,
be resistant to noisy features,
ability to scale in feature dimension,
deal with multiple classes and performing detection efficiently (i.e. many
autonomous systems require real-time or close to real-time performance).

\subsection{Review}
Many approaches have been made based on statistical methods. Those often model
the training data with statistic distributions and then perform hypothesis tests
For example performing checking whether two distributions are the same,
distance to mean expected value or quartiles, etc\dots
In some sense all they define a distance measure to normality and employ a
threshold on that.

A common approach is to use density estimation and use the expected probability
of a sample to trigger the sample as novel. Examples are usage of
Gaussian Mixture Models and Parzen-window estimators. In order to be effective
those use data as close as possible to the input features.
And dimension-reduction techniques such as \gls{PCA} are used to make density
estimation more feasible. Note that as dimension increases an exponential number
of data samples are required to approach the density with a given quality.

\cite{bishop1994novelty} uses that approach by employing a Parzen-window to
estimate the density of the training data on a given input feed into a
neural-network. By triggering on the calculated density they detect samples
that differ from the training data and consider their neural-network output
to be unreliable as the samples are distinct from what the network was trained
with.

A slightly different approach is done by one-class \gls{SVM} approaches that
try to distinguish the class by separating the traing set from all the other
points in the input space. They try to achieve that by enclosing the training
set by some structure (i.e. an hyper-sphere)~\cite{bennett2000support}.
Those approaches have been made in line with Vapnik principle of not solving
something hard. As although having access to a perfect probability distribution
of the input would solve the problem, creating such a function is harder than
simply creating a boundary between known data and novel data
\cite{scholkopf2000support}.

Error reconstruction methods have also been used for novelty detection.
They use the assumption that the class to be defined lies on a manifold embedded
on the input space of higher dimensions. By using dimensional reduction
techniques, they try to defined that manifold and calculate how far a new sample
is from it.
One of the most common methods used for that is
\gls{K-PCA}~\cite{scholkopf1997kernel}, which uses the kernel-trick to extend
\gls{PCA} and perform a nonlinear dimensionality reduction on the input.
This technique has been successfully used in novelty detection by
\cite{Hoffmann2007863}.

\subsection{Related Work}
More in line with novelty detection in scene categorization and semantic
structures, the work of \gls{PLISS}~\cite{ranganathan2010pliss}.\dots 



\section{Novelty Detection as a Threshold}

% Explain threshold approach to novelty detection
Due to the desire of robustness, a novelty detection system is often given the option
to classify non-novel samples as novel. This is implemented as a threshold that describes a trade
between error and rejection: the more the system tries to reduce rejection, the more
prone to produce errors it becomes.

When dealing with a statistic point of view, noisy data or unstable features,
a decision to classify a sample $x$ as novel reduces rejection rate by
$P(novel|x)P(x)$ by introducing an associated error probability of $P(\overline{novel}|x)P(x)$.
Generalizing to a system that classifies a set of inputs $N$ as novel, the
true positive and false negative rates arise:

\begin{equation}
\label{eq:true-positive}
P(\textnormal{true positive}) = \sum_{x \in N}{P(novel|x)P(x)}
\end{equation}

\begin{equation}
\label{eq:false-positive}
P(\textnormal{false positive}) = \sum_{x \in N}{P(\overline{novel}|x)P(x)}
\end{equation}

The performance of a novelty detection system can be measured on these two
rates: the true-positive rate (\autoref{eq:true-positive}) describes the
interest in detecting as much as possible the novel cases;
and the false-positive rate (\autoref{eq:false-positive}) the interest in
having the smallest error possible.
As a given sample $x$ is included in $N$ both the true-positive and
false-positive rates increase or stay equal\footnote{
Note that both are described as summations of products of probabilities
(non-negative functions).} describing the trade-of any detection system
has to consider: by increasing it detection rate it also has to increase
the error rate.

An optimal novelty detector can then be formulated as selecting a subset $N$
that achieves a given true-positive rate with the minimal false-positive
rate possible. This is equivalent to the \emph{continuous knapsack problem}
which allows a greedy solution by sorting the items by value per weight.
Therefore a novelty system before classifying a sample $a$ as novel should
classify any sample $b$ satisfying \autoref{eq:knapsack} as that would achieve
a lower false-positive rate given a fixed true-positive rate.

\begin{equation}
\label{eq:knapsack}
\frac{P(\overline{novel}|b)P(b)}{P(novel|b)P(b)} < \frac{P(\overline{novel}|a)P(a)}{P(novel|a)P(a)}
\end{equation}

This relation between $a$ and $b$ can then be simplified into:

\begin{equation}
P(\overline{novel}|b) < P(\overline{novel}|a)
\end{equation}


Based on this, it can be said that an optimal novelty detection system is
interested in defining an order relation on all the possible inputs equivalent
to the order defined by the error rate: $P(\overline{novel}|x)$.
And any optimal detector can be described by the largest $P(\overline{novel}|x)$
accepted by it. Which is seen as threshold.


\section{Conditional and Unconditional Probability Ratio}

On the previous section it was shown that an optimal novelty detector can be
implemented with a threshold on top of the order-relation defined by
$P(\overline{novel}|x)$ over $x$. Performing some manipulations with
Bayes theorem and assuming a constant $P(\overline{novel})$ a more usable
form can be attained:

\begin{equation}
\label{eq:novelty-ratio}
          P(\overline{novel}|x)
  =       \frac{P(x|\overline{novel}) P(\overline{novel})}{P(x)}
  \propto \frac{P(x|\overline{novel})}{P(x)}
\end{equation}

Since there is only interest in maintaining the same order-relation as
$P(\overline{novel}|x)$ any constant factor can be dropped.
Leaving a ratio, between a \emph{conditional} and
\emph{unconditional probability}, suitable for implementing novelty detection
via thresholding.


\subsection{Conditional Probability}
The conditional probability $P(x|\overline{novel})$ describes the distribution
of the non-novel samples on the original distribution. In case the labelled data
available for the agent to learn a concept comes from the same distribution
where the system will run the correct approach is to use it as prior-information
for modelling the conditional probability.

Note that it is important for the labelled data to come from that same
distribution, otherwise it will contain some bias and lead to an incorrect
ordering. The bias, for example, can be introduced by an incorrect filtering or
data balancing.

\subsection{Unconditional Probability}
The unconditional probability $P(x)$ plays an important role on obtaining a
correct order relation for performing novelty detection.
It serves as a normalizing component that allows the system to figure out
whether a given sample conditional probability arises from it belonging to
the known concept or from the likelihood of being sampled.


On lack of any information about the unconditional probability and conforming to
the principle of maximum entropy (\autoref{sec:max-entropy}) a uniform
distribution must be chosen.
Though, as discussed in \autoref{sec:unlabelled-data}, by using unlabelled data
it becomes possible to obtain prior-information and achieve a better
approximation.


%\subsection{Novelty Detection on a Variable Set of Features}
Note also that often novelty detection is applied on a fixed set of features
together with an assumption of a uniform unconditional probability.
On those cases $P(x)$ becomes a constant and therefore a novelty threshold
can be directly applied on $P(x|\overline{novel})$ as is the case in \cite{bishop1994novelty}.
But in the case where the set of features $x$ is variable it cannot be
discarded. There $P(x)$ also plays a role in levering all the conditional
probabilities on different sets of variables into the same measure units
such that a threshold can be implemented.



\section{A Practical Example}
\label{sec:unlabelled-data}
In order to sum up the presented concepts on novelty detection with a threshold
function and exemplify how to use graphical models in the context of
multi-modality room classification a synthetic dataset was generated.
The dataset was kept simple by only modelling directly sensed features from a
room, skipping any structural knowledge or sensing model such as room
connectivity and extra hidden variables.

In this dataset a room $r$ is seen as an hidden-variable generator of a set of
features $X$ that are directly sensed by the agent.
All the sensed features $x$ are all independent given the room category.
In whole there was 11 different room categories and 7 different feature types.
Each feature can be sensed more than once (i.e. room shape is extracted from 2D
laser scans in more than one position in the room), but all those sensed
instances are considered independent given the room category.

The room categories were chosen to mimic as close as possible the real features
and categories existing in reality (i.e. 1 person office, 2 person office,
hallway, robot lab, etc\dots). A table describing the used synthetic
distribution is given in \autoref{extra:synthetic-distribution}.

The objective then was to draw a system that, although only trained with
labelled data from 5 of the 11 room categories, was able to detect novel
room categories.
For that 100 labelled samples for the 5 known categories were drawn and 1000 
unlabelled samples were drawn from all the room categories for learning the
unconditional probability distribution and measure effect of using unlabelled
data.

\subsection{Conditional Probability}
Using the labelled samples, 7 factors $\phi_X(r,x)$ were created, one for each
feature type, to represent the potential of sensing features $x$ of type $X$ on
a room of category $r$.

\begin{equation}
\phi_X(r,x) = number of samples + C
\end{equation}

With those the probability of sensing a set of features $x$ on a room knowing
room category is known to the agent can be modelled with a factor graph as
illustrated on \autoref{fig:simple-cond-graph}.

\begin{figure}[h]
\centering
\begin{tikzpicture}
  \node [matrix,matrix anchor=mid, column sep=20pt, row sep=10pt,ampersand replacement=\&] {
    \& \& \node (room) [latent] {$r$}; \& \& \\
    \& \& \& \& \\
    \node (f1) [factor] {}; \&
    \node (f2) [factor] {}; \&
    \node (f3) [factor] {}; \&
    \node (fi) [] {\dots}; \&
    \node (fn) [factor] {}; \\
    \node (x1) [obs] {$x_1$}; \&
    \node (x2) [obs] {$x_2$}; \&
    \node (x3) [obs] {$x_3$}; \&
    \node (xi) [] {\dots}; \&
    \node (xn) [obs] {$x_n$}; \\
  };
  \draw [-] (room) -- (f1) -- (x1);
  \draw [-] (room) -- (f2) -- (x2);
  \draw [-] (room) -- (f3) -- (x3);
  \draw [-] (room) -- (fi);
  \draw [-] (room) -- (fn) -- (xn);

  \node (captf1) [right=2pt of f1] {\footnotesize{$\phi_{X_1}$}};
  \node (captf2) [right=2pt of f2] {\footnotesize{$\phi_{X_2}$}};
  \node (captf3) [right=2pt of f3] {\footnotesize{$\phi_{X_3}$}};
  \node (captfn) [right=2pt of fn] {\footnotesize{$\phi_{X_n}$}};
\end{tikzpicture}
\caption{\label{fig:simple-cond-graph}A factor graph modelling the conditional
probability of sensing a set of features $x$ known that the room category $r$ is
one of the known classes.}
\end{figure}

\subsection{Unconditional Probability}
With no knowledge on the unconditional probability the correct approach is to
assume a uniform distribution.
That is represented in factor graphs by a graph without any factors.

\begin{figure}[h]
\centering
\begin{tikzpicture}
  \node [matrix,matrix anchor=mid, column sep=20pt, row sep=10pt,ampersand replacement=\&] {
    \node (x1) [obs] {$x_1$}; \&
    \node (x2) [obs] {$x_2$}; \&
    \node (x3) [obs] {$x_3$}; \&
    \node (xi) [] {\dots}; \&
    \node (xn) [obs] {$x_n$}; \\
  };
\end{tikzpicture}
\caption{\label{fig:simple-uniform-graph}A factor graph modelling a uniform
         distribution over the sensed set of features $x$.}
\end{figure}

Very often there is extra knowledge that can be obtained about the distribution
of the variable that helps to model the unconditional distribution.
In this practical example the access to unlabelled data is explored as access to
it is very common on applications in robotics.

Note that the sensed variables are dependent between each other when the room
category $r$ is not known. With that the correct approach would be to train a
factor that is able to correlate all the sensed variables as seen on
\autoref{fig:simple-all-dependent}.

\begin{figure}[h]
\centering
\begin{tikzpicture}
  \node [matrix,matrix anchor=mid, column sep=20pt, row sep=10pt,ampersand replacement=\&] {
    \& \& \node (fac) [factor] {}; \& \& \\
    \& \& \& \& \\
    \node (x1) [obs] {$x_1$}; \&
    \node (x2) [obs] {$x_2$}; \&
    \node (x3) [obs] {$x_3$}; \&
    \node (xi) [] {\dots}; \&
    \node (xn) [obs] {$x_n$}; \\
  };
  \draw [-] (fac) -- (x1);
  \draw [-] (fac) -- (x2);
  \draw [-] (fac) -- (x3);
  \draw [-] (fac) -- (xn);
  \draw [-] (fac) -- (xi);
\end{tikzpicture}
\caption{\label{fig:simple-all-dependent}A general factor graph able
         to model any unconditional distribution on the sensed variables
         requires a factor connecting all of them.}
\end{figure}

Nonetheless such an approach suffers from
the \emph{curse of dimensionality} as the number of sensed features and feature
types increases.

A relaxed solution can be obtained by assuming in that the sensed features are
independent conditional 
Although the sensed variables are considered independent given the room category
$r$, when $r$ is not able to model  hidden
variable $r$. The approach should then be to 
In order to correctly approach the unconditional probability the unlabelled
data should be used 
Without the assumption of the room category being known it is no longer easy
to model the distribution between all sensed variables.
Under the assumption 

Note that although sensed variables are independent, they are only so when $r$
is given. Requiring a factor to be connect between all variables to model their
dependence.

\begin{figure}[h]
\centering
\begin{tikzpicture}
  \node [matrix,matrix anchor=mid, column sep=20pt, row sep=10pt,ampersand replacement=\&] {
    \& \& \& \& \\
    \node (f1) [factor] {}; \&
    \node (f2) [factor] {}; \&
    \node (f3) [factor] {}; \&
    \node (fi) [] {\dots}; \&
    \node (fn) [factor] {}; \\
    \node (x1) [obs] {$x_1$}; \&
    \node (x2) [obs] {$x_2$}; \&
    \node (x3) [obs] {$x_3$}; \&
    \node (xi) [] {\dots}; \&
    \node (xn) [obs] {$x_n$}; \\
  };
  \draw [-] (f1) -- (x1);
  \draw [-] (f2) -- (x2);
  \draw [-] (f3) -- (x3);
  \draw [-] (fi);
  \draw [-] (fn) -- (xn);
\end{tikzpicture}
\caption{\label{fig:simple-independent-graph}A factor graph modelling an
         independent distribution over the sensed set of features $x$.}
\end{figure}



\subsection{Threshold Functions}
Three threshold functions were created using the knowledge on the synthetic
distribution and the models learnt from the sample data:
$G$ on \autoref{fig:simple-cond-graph},
$U$ on \autoref{fig:simple-uniform-graph} and
$I$ on \autoref{fig:simple-independent-graph}.

\begin{description}
\item[exact]
$P(x|\overline{novel})/P(x)$
Since the distribution is synthetic there is access to $P(x)$ and $P(x|concept)$
and a perfect threshold function could also be created to test how far the
presented thresholds are from optimal.

\item[uniform]
$P_G(x)/P_U(x)$ assuming a uniform unconditional distribution

\item[independent]
$P_G(x)/P_I(x)$ assuming an independent unconditional distribution.
\end{description}


\subsection{Probability Ratio Comparison}
%%% Results 1
% Show the threshold ratio is an optimal detector (if perfect information was available)
% Show that the thresholds are suitable functions for implementing a static threshold.
As first tryout, the performance of the novelty threshold selection was plotted for a set
of 1000 samples taken out from the whole distribution (\autoref{fig:synthetic-roc}).
Those samples contain different number of sensed properties for each room, mimicking
the dynamic properties expected to see when implemented on a robot.

\begin{figure}[h]
\centering
\includegraphics[width=0.60\textwidth]{results/synthetic-all.pdf}

\caption{\label{fig:synthetic-roc}ROC curve comparing novelty detection performance
         under samples with variable size of sensed properties.}
\end{figure}

The convex shape for the optimal threshold shows that the ratio between conditional
and unconditional probability is indeed an optimal detector and is suitable for
implementing a threshold when the samples are taken from dynamic
distributions (e.g.: some samples where there is only access to room size versus
samples where there is a lot of information about the room properties).

% Discuss importance on approximating unconditional probability.
Its also possible to see how important it is to estimate a correct unconditional
probability in order to obtain a correct novelty measure on the inputs.
The assumption of a uniform unconditional probability has led to very poor results.
That is probably explained by the semantic properties being highly
biased towards some values. And shows that bias plays an important step
in detecting whether a given sensed value is a valuable cue about the room category.



%%% Results 2
% Measure performance of the thresholds as more information becomes available.
\subsection{Performance Changes With Amount of Available Information}
In order to measure the performance impact as more semantic information becomes
available ROC curves were plotted for samples grouped by the number of sensed
semantic features.

\begin{figure}[h]
\centering

\subfloat[3 sensed features]{\includegraphics[width=0.40\textwidth]{results/synthetic-3features.pdf}}
\qquad
\subfloat[5 sensed features]{\includegraphics[width=0.40\textwidth]{results/synthetic-5features.pdf}}

\subfloat[10 sensed features]{\includegraphics[width=0.40\textwidth]{results/synthetic-10features.pdf}}
\qquad
\subfloat[50 sensed features]{\includegraphics[width=0.40\textwidth]{results/synthetic-50features.pdf}}

\caption{\label{fig:synthetic-roc-breakdown}ROC curves plotted showing performance of the
         presented novelty detection method on graphs generated for different amount of
         sensed features.}
\end{figure}

It is possible to see that as the system gains more semantic information it
becomes easier to detect novelty. The input space size increases and allows the
several existing classes to become more easily distinguished.

The performance of the independent threshold decreases as the number of sensed
features increases. This is easily explained by the fact that the graph $I$ is not
able to model the existent dependence between the features. This becomes obvious
as the number of features increases (e.g.: graph $I$ perfectly models $P(x)$ in the
case where only 1 feature is sensed).

The uniform threshold shows a poor performance specially on small size samples
where its performs almost no better than random.
It performance increases as the size of sensed features increases but nonetheless
its very small when compared to how optimal a threshold could be.



