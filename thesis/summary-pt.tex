%-----------------------------------------------
% Template para criação de resumos de projectos/dissertação
% jlopes AT fe.up.pt,   Fri Jul  3 11:08:59 2009
%-----------------------------------------------

\documentclass[9pt,a4paper]{extarticle}

%% English version: comment first, uncomment second
\usepackage[portuguese]{babel}  % Portuguese
%\usepackage[english]{babel}     % English
\usepackage{graphicx}           % images .png or .pdf w/ pdflatex OR .eps w/ latex
\usepackage{times}              % use Times type-1 fonts
\usepackage[utf8]{inputenc}     % 8 bits using UTF-8
\usepackage{url}                % URLs
\usepackage{multicol}           % twocolumn, etc
\usepackage{float}              % improve figures & tables floating
\usepackage[tableposition=top]{caption} % captions
%% English version: comment first (maybe)
\usepackage{indentfirst}        % portuguese standard for paragraphs
%\usepackage{parskip}

%% page layout
\usepackage[a4paper,margin=30mm,noheadfoot]{geometry}

%% space between columns
\columnsep 12mm

%% headers & footers
\pagestyle{empty}

%% figure & table caption
\captionsetup{figurename=Fig.,tablename=Tab.,labelsep=endash,font=bf,skip=.5\baselineskip}

%% heading
\makeatletter
\renewcommand*{\@seccntformat}[1]{%
  \csname the#1\endcsname.\quad
}
\makeatother

%% avoid widows and orphans
\clubpenalty=300
\widowpenalty=300

% Additional packages
\usepackage[utf8]{inputenc}
\usepackage{subfig}
\usepackage[pdftex,pdfpagelabels,bookmarks,hyperindex,hyperfigures]{hyperref}
\hypersetup{%
   plainpages=false, 
   pdfpagelayout=SinglePage,
   bookmarksopen=false,
   bookmarksnumbered=true,
   breaklinks=true,
   linktocpage,
   colorlinks=true,
   linkcolor=blue,
   urlcolor=blue,
   citecolor=blue,
   anchorcolor=green
}      



\begin{document}

\title{\vspace*{-8mm}\textbf{\textsc{Novelty Detection for Semantic Place Categorization}}}
\author{\emph{André Susano Pinto}$^\dag$\\[2mm]
        \small{Sob orientação de: Andrzej Pronobis$^\ddag$, Luis Paulo Reis$^\dag$}\\
        \small{$^\dag$Faculdade de Engenharia da Universidade do Porto, Portugal}\\
        \small{$^\ddag$Royal Institute of Technology (KTH), Stockholm, Sweden}}
%\url{andresusanopinto@gmail.com}\\
\date{}
\maketitle
%no page number 
\thispagestyle{empty}

\vspace*{-4mm}\noindent\rule{\textwidth}{0.4pt}\vspace*{4mm}

\begin{multicols}{2}

\section{Motivation}
Já à muitos anos que a humanidade sonha que um dia robôs caminharam lado a lado com os humanos.
Acredita-se que eles serão capaz de ser mover e interagir connosco, compreender os nossos conceitos
e serem capazes de pensar.
Um passo importante nesta direcção é a criação de robôs com conhecimento de conceitos e significados
próprios do mundo humano.
No entanto é irrealista acreditar que será possivel descrever e representar em toda a sua extensão
tais conceitos. Assim os agentes deverão ser capazes de aprender e adaptar o seu conhecimento quando
confrontados com novas situações.
A detecção destas novas situações onde o conhecimento de um agente não é suficiente têm um papel importante
na adaptação e aprendizagem de novos conceitos e significados e é por isso o tópico principal desta tese.

Addicionalmente no contêxto de agentes moveis, para facilitar o deslocamento e planeamento, os
agentes constroiem uma representação do ambiente na qual detectam conceitos e propriedades
conceitos espaciais.
Essa representação identifica conceitos e classifica-os de acordo com significados humanos,
permitindo assim ao agente obter uma representação dos conceitos humanos e pensar e planear a um
nível próximo dos mesmos.
Ao processo de construção desta representação é attribuido o nome de \emph{semantic mapping}.
Por exemplo, a uma escala de quartos, os humanos categorizam o espaço em: cozinhas, quartos, corredores,
escritórios, sala de computadores, etc\dots
Ao relacionar quartos com estas categorias que descrevem propriedades e acções percebidas por
humanos, um agente consegue resolver tarefas como: ``Onde estão os ceriais?''

\section{Problema e Objectivos}
Como deliniado na motivação, o processo de \emph{semantic mapping} têm um papel importante na
capacidade pensativa de um agente. Em simultâneo, esse processo está limitado pelo conhecimento
do agente em conceitos e significados do mundo humano. É então importante estudar e analisar
técnicas que sejam capazes de detectar falhas no conhecimento e permitir ao agente agir de maneira
cuidadosa.
É com esse objectivo que a tese aqui descrita se foca no estudo de deteção
de situações novas num processo de \emph{semantic mapping} propôsto por \cite{pronobis2011phd}.
Para esse fim a tese estuda como usar modelos graphicos para detectar novas categorias nas conhecimento
de categorias do agente.

\section{Mapeamento Semântico}
Mapeamento semântico é o processo através do qual um agente constroi uma representação do ambiente
extendida com conceitos e categories com significado especial (i.e.\ conceitos humanos).
Neste tese o processo proposto por \cite{pronobis2011phd} foi usado como base.

\subsection{Conhecimento Espacial}
O conhecimento espacial de agente define todo a informação necessária para o mapeamento semântico,
e inclui informação em como segmentar e identificar entidades e propriedades através dos dados
recolhidos.
Para esse fim Pronobis~\cite{pronobis2011semmap} utiliza uma representação estruturada em camadas:
a \emph{camada de sensores} é responsável por manter uma descrição exacta do ambiente numa representação
de baixo nível;
a \emph{camada de lugares} é responsável por segmentar e identificar areas e conceitos
como quartos;
a \emph{camada de categories} é responsável por definir conhecimento sobre propriadades e categories
e permite ao agente obter informação de mais alto-nível;
e a \emph{camada de conceitos} mantêm conhicimento sobre os diversoes conceitos, relações sobre os mesmos
e informação de como os mapear para conceitos humanos.

\subsection{Conceptual Map}
The semantic mapping is then based on the instantiation of a conceptual map that uses all the
ontology defined on the conceptual layer in order to build a probabilistic graphical model that
relates and propagates probabilities across all the entities and properties detected by the
other layers.

By using probabilistic graphical models, the conceptual map becomes then a generative model that
allows to perform inference on variable configurations. For that it is used as a base tool
for novelty detection of unknown categories on some of its variables.

\section{Novelty Detection}
Novelty detection, also known as outlier or anomaly detection, is a
classification problem related to identification of new or unknown data
patterns that the system is not aware of~\cite{markou2003novelty}.
The ability to identify novel cases is crucial in any autonomous system
that is deployed to unknown or to uncontrolled environments, as it gives the
system the ability to detect that something is not conforming to its knowledge and
therefore should be treated with caution.
% It has several applications such as fault detection~\cite{tarassenko1999novelty},
% intrusion detection~\cite{fan2001using},
% detection of masses in mammograms~\cite{tarassenko1995novelty} or detection of
% novel and useful documents~\cite{zhang2002novelty}.

\subsection{Novelty Detection as Threshold}
Due to the noisy, uncertain and insufficient nature of sensed data, novelty ought to be
treated in a probabilistic way in where each sample has a certain probability
of being generated by a class not known for the agent and a complementary probability $P(\overline{novel}|x)$
of being generated by a known class.

Additionally any detector can be uniquely described by the set $N$ of samples that it
classifies as novel. With that its possible to show that by including a new sample $a$ in $N$ an agent
can increase its detection rate at a cost of producing more mis-classifications.
This describes the base of the \emph{error and rejection tradeoff} described by \cite{chow1970optimum}.
Due to the interest of having the highest detection rate with a given minimum error rate, any optimal
detector can be described as a \emph{continuous knapsack problem} of the sample space.
The greedy ordering defined to solve it can be shown to be equivalent to:
\begin{eqnarray}
P(\overline{novel}|x) &=& \frac{P(x|\overline{novel})P(\overline{novel})}{P(x)}
\end{eqnarray}

\subsection{Assumption on a constant $P(novel)$}
By considering $P(novel)$ to be constant a ratio between conditional and unconditional probabilities
of the sensed variables is obtained.
Although it sounds reasonable to assume the probability of specific room belong to a novel
category is constant, this thesis points that this assumption might be very strong and not
reasonable in a realistic scenario. The main point being that in this case the graph structure
changes between samples and some might be more prone to produce novel samples, as opposed to normal
classification scenarios where its clear that the category is indeed the only generator of features.

\subsection{Modeling Conditional Probability}
Since the graphical model produced by the conceptual map tries to model the distribution of
variables assuming the knowledge of the agent holds true, it represents a method to model
$P(x|\overline{novel})$.
%since it allows to calculate the density probability that the set of features $x$ is
%sensed given that all the variables and graph structure, including the category of
%a specific room $a$ are correctly modelled by the agent knowledge.
%This way the distribution modelled by a factor graph equivalent to the chain-graph used by
%the conceptual map can be used as an approximation for the conditional probability of $x$.

\subsection{Modeling Unconditional Probability}
With no knowledge on the real distribution samples, the correct approach is to model it
with a uniform distribution. Nonetheless in the presented case, it was assumed that only the room
category of a specific room is unknown and that the remaining knowledge holds trues.
In that case, the correct approach is to replace the variable that represents the room where to
detect novelty with a single factor connecting all variables directly dependent on it. 

Due to its high-dimensionality, approximating that factor with unlabelled data is infeasible and
approximations have to be done. In this thesis two approximations were presented: an uniform model
and an independent model. In the independent model, the
single\hyp{}connected factors are trained from unlabelled data and its expected that they create a better
approximation to the unconditional probability than an assumption on the uniform model.

\section{Results}
A synthetic dataset modelling the expected features when implementing on the real case and enforcing
a constant $P(novel)$ was used for testing the developed models.
The two suggested models where compared between them and to an optimal case.
The tests across several situations shows how they behaviour change as more information is sensed
from the environment.

\begin{figure}[H]
\centering
\includegraphics[width=0.4\textwidth]{results/synthetic-all.pdf}
\caption{\label{fig:results}Comparison of the proposed models using samples from a synthetic dataset
         with variable number of sensed features.}
\end{figure}

\section{Conclusions}
This thesis studied the problem of novelty detection of semantic categories on the spatial knowledge
of mobile robots. To that purpose a semantic mapping process~\cite{pronobis2011semmap} was studied and
a novelty detection scheme based on the conceptual map used by it proposed.
Initial tests where performed on synthetic data showing an interest in using unlabelled data to
increase the detection performance.
Additionally all the assumptions and bases behind novelty detection were presented and several
directions on how to create a generalized framework for novelty detection were discussed.


\bibliographystyle{unsrt}
\bibliography{refs}

\end{multicols}

\end{document}

