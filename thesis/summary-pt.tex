%-----------------------------------------------
% Template para criação de resumos de projectos/dissertação
% jlopes AT fe.up.pt,   Fri Jul  3 11:08:59 2009
%-----------------------------------------------

\documentclass[9pt,a4paper]{extarticle}

%% English version: comment first, uncomment second
\usepackage[portuguese]{babel}  % Portuguese
%\usepackage[english]{babel}     % English
\usepackage{graphicx}           % images .png or .pdf w/ pdflatex OR .eps w/ latex
\usepackage{times}              % use Times type-1 fonts
\usepackage[utf8]{inputenc}     % 8 bits using UTF-8
\usepackage{url}                % URLs
\usepackage{multicol}           % twocolumn, etc
\usepackage{float}              % improve figures & tables floating
\usepackage[tableposition=top]{caption} % captions
%% English version: comment first (maybe)
\usepackage{indentfirst}        % portuguese standard for paragraphs
%\usepackage{parskip}

%% page layout
\usepackage[a4paper,margin=30mm,noheadfoot]{geometry}

%% space between columns
\columnsep 12mm

%% headers & footers
\pagestyle{empty}

%% figure & table caption
\captionsetup{figurename=Fig.,tablename=Tab.,labelsep=endash,font=bf,skip=.5\baselineskip}

%% heading
\makeatletter
\renewcommand*{\@seccntformat}[1]{%
  \csname the#1\endcsname.\quad
}
\makeatother

%% avoid widows and orphans
\clubpenalty=300
\widowpenalty=300

% Additional packages
\usepackage[utf8]{inputenc}
\usepackage{subfig}
\usepackage[pdftex,pdfpagelabels,bookmarks,hyperindex,hyperfigures]{hyperref}
\hypersetup{%
   plainpages=false, 
   pdfpagelayout=SinglePage,
   bookmarksopen=false,
   bookmarksnumbered=true,
   breaklinks=true,
   linktocpage,
   colorlinks=true,
   linkcolor=blue,
   urlcolor=blue,
   citecolor=blue,
   anchorcolor=green
}      



\begin{document}

\title{\vspace*{-8mm}\textbf{\textsc{Novelty Detection for Semantic Place Categorization}}}
\author{\emph{André Susano Pinto}$^\dag$\\[2mm]
        \small{Sob orientação de: Andrzej Pronobis$^\ddag$, Luis Paulo Reis$^\dag$}\\
        \small{$^\dag$Faculdade de Engenharia da Universidade do Porto, Portugal}\\
        \small{$^\ddag$Royal Institute of Technology (KTH), Stockholm, Sweden}}
%\url{andresusanopinto@gmail.com}\\
\date{}
\maketitle
%no page number 
\thispagestyle{empty}

\vspace*{-4mm}\noindent\rule{\textwidth}{0.4pt}\vspace*{4mm}

\begin{multicols}{2}

\section{Motivação}
Já à muitos anos que a humanidade sonha que um dia robôs caminharam lado a lado com os humanos.
Acredita-se que eles serão capaz de se mover e interagir connosco, compreender os nossos conceitos
e serem capazes de pensar.
Um passo importante nesta direcção é a criação de robôs com conhecimento de conceitos e significados
próprios do mundo humano.
No entanto é irrealista acreditar que será possível descrever e representar em toda a sua extensão
tais conceitos. Assim os agentes deverão ser capazes de aprender e adaptar o seu conhecimento quando
confrontados com novas situações.
A detecção destas novas situações, onde o conhecimento de um agente não é suficiente, têm um papel importante
na adaptação e aprendizagem de novos conceitos e significados e é por isso o tópico principal desta tese.

No contexto de agentes moveis, para facilitar o deslocamento e planeamento, os
agentes constroem uma representação do ambiente. Quando essa representação é estendida com
conceitos e categorias com significado para humanos, as tarefas de alto nível ficam mais fáceis.
Por exemplo, a uma escala de quartos, os humanos categorizam o espaço em: cozinhas, quartos, corredores,
escritórios, etc\dots
Ao relacionar quartos com estas categorias que descrevem propriedades e acções percebidas por
humanos, um agente consegue resolver tarefas como: ``Onde estão os cereais?''

\section{Problema e Objectivos}
O processo de mapeamento semântico têm um papel importante na capacidade pensativa de um agente.
Em simultâneo, esse processo está limitado pelo conhecimento em conceitos e significados do mundo humano.
É então importante estudar e analisar técnicas que sejam capazes de detectar falhas no conhecimento e permitir ao agente agir de maneira
cuidadosa.
É com esse objectivo que a tese aqui descrita se foca no estudo de técnicas de detecção de situações novas
e de como as aplicar num processo de mapeamento semântico com vista a detectar novas categorias de lugares
em ambientes interiores. Para esse fim, a tese estuda o processo de mapeamento semântico proposto por \cite{pronobis2011phd},
e procura solucionar a detecção de novas categorias usando modelos gráficos.

\section{Mapeamento Semântico}
Mapeamento semântico é o processo através do qual um agente constrói uma representação do ambiente
estendida com conceitos e categorias com significado especial (i.e.\ conceitos humanos).
Neste tese o processo proposto por \cite{pronobis2011phd} foi usado como base.

\subsection{Conhecimento Espacial}
O conhecimento espacial de agente define a informação necessária para o mapeamento semântico,
e inclui informação em como segmentar e identificar entidades e propriedades através dos dados
recolhidos.
Para esse fim \cite{pronobis2011semmap} utiliza uma representação estruturada em camadas:
a \emph{camada de sensores} mantém uma descrição exacta do ambiente numa representação
de baixo nível;
a \emph{camada de lugares} segmenta e identifica áreas e conceitos como quartos;
a \emph{camada de categorias} é responsável por definir conhecimento sobre propriedades e categorias
e permite ao agente obter informação de mais alto nível;
e a \emph{camada de conceitos} mantêm conhecimento sobre os diversos conceitos, relações entre os mesmos
e informação de como os mapear para conceitos humanos.

\subsection{Mapa Conceptual}
Durante o processo de mapeamento semântico o agente utiliza a ontologia da camada conceptual para produzir
um mapa conceptual. Este é produzido usando modelos gráficos probabilísticos que relacionam
os conceitos detectados e propagam probabilidades entre os mesmos de acordo com o conhecimento do agente.
A detecção de uma nova categoria no conhecimento do agente traduz-se na detecção de um novo estado nas
variáveis de tal grafo e portanto os mapas conceptuais são usados como ferramenta para detecção de novas
categorias.

\section{Novelty Detection}
\emph{Novelty detection}, também conhecido por \emph{detecção de anomalias} ou \emph{detecção de outliers},
é o problema de detecção que os dados foram produzidos por uma nova classe,
sobre a qual o agente não têm conhecimento~\cite{markou2003novelty}.
A habilidade de detectar estes casos é de fulcral importância em sistemas autónomos que operam em ambientes
desconhecidos. Pois permite detectar que algo está fora do normal e assim actuar com cuidado.

\subsection{\emph{Novelty Detection} através de \emph{Thresholds}}
Devido a ruídos e incerteza existentes nos dados, a detecção de casos novos é tratada com
probabilidades. Assim para cada amostra de dados $x$ existe uma certa probabilidade de a mesma
ser gerada por uma classe desconhecida e uma probabilidade complementar $P(\overline{novel}|x)$
de ser por uma classe conhecida.

Além disso, qualquer detector por ser descrito unicamente pelo conjunto $N$ de amostras que classifica
como novas. Com isso é possível demonstrar que ao expandir o conjunto $N$ com um novo elemento $a$,
o agente aumenta a taxa de detecção a um custo de produzir mais classificações erradas.
Isto descreve o principio de troca entre erro e rejeição descrito por \cite{chow1970optimum}.
Um detector óptimo, é assim equivalente ao problema da mochila com quantidades fraccionarias, pois têm
como objectivo obter a maior taxa de detecção com uma taxa de erro máximas. Assim qualquer detector
óptimo pode ser descrito através de um limite imposto sobre a relação gulosa dado por:
\begin{eqnarray}
P(\overline{novel}|x) &=& \frac{P(x|\overline{novel})P(\overline{novel})}{P(x)}
\end{eqnarray}

\subsection{Critério: $P(novel)$ como Constante}
Ao considerar $P(novel)$ constante, uma relação equivalente pode ser obtida usando somente a fracção entre $P(x|\overline{novel})$ e $P(x)$.
Apesar de parecer inofensivo, o caso apresentado neste tese é diferente dos casos normais em que todos as amostras
são geradas através de uma categoria. Aqui toda estrutura e relação entre as variáveis toma um papel na geração
de amostras e portanto não é claro como tal critério se comporta na realidade.

\subsection{Modelar $P(x|\overline{novel})$}
O modelo gráfico produzido durante o mapeamento semântico, modela a distribuição de todas as variáveis
assumindo que o conhecimento do agente é representativo da realidade, como tal pode ser usado
para aproximar $P(x|\overline{novel})$.

\subsection{Modelar $P(x)$}
Sem conhecimento sobre a distribuição real é correcto modela-la usando uma distribuição
uniforme. No entanto, o agente considera que todo o seu conhecimento é representativo
da realidade à excepção da categoria de uma das suas variáveis. Nesse caso a aproximação correcta é substituir
os factores dependentes dessa variável por um único factor.
Devido à dimensão do mesmo, é inviável aproximá-lo sem realizar simplificações.
A tese apresenta assim duas aproximações: um modelo uniforme e um modelo independente. No modelo independente
esse factor é aproximado com vários factores dependentes somente numa variável. Esses podem facilmente ser
aproximados e é esperado que obtenham melhores resultados ao compensar por uma tendência do ambiente em gerar
certas \emph{features}.

\section{Resultados}
Para testes um conjunto de dados sintético foi preparado. Este aproxima as propriedades
esperadas num cenário realístico como por exemplo um número variável de \emph{features}.
Os modelos sugeridos foram comparados entre eles e entre um detector óptimo, e os testes realizados
mostram o comportamento dos modelos em diversas situações, por exemplo conforme mais informação
é recolhida pelo agente.

\begin{figure}[H]
\centering
\includegraphics[width=0.4\textwidth]{results/synthetic-all.pdf}
\caption{\label{fig:results}Comparação dos modelos propostos usando dados sintéticos com número variável de \emph{features}.}
\end{figure}

\section{Conclusão}
Esta tese estudou o problema da detecção de novas categorias de espaços interiores no contexto de agentes moveis.
Para tal o processo de mapeamento semântico~\cite{pronobis2011semmap} foi estudado e um mecanismo de detecção,
baseado no mapa conceptual usado pelo mesmo, foi proposto.
Foram realizados testes iniciais com dados sintéticos, mostrando o benefício de usar dados não supervisionados.
Adicionalmente toda a teoria e assunções por trás do método apresentado foram documentadas, incluindo
discussão de diversas direcções futuras em como criar uma \emph{framework} generalizada para detecção de casos
novos no conhecimento semântico e espacial de agentes.

\bibliographystyle{unsrt}
\bibliography{refs}


\end{multicols}

\end{document}

