\chapter{Introduction}
% Introduce:
%  Mobile Robotics, A.I.
%  Interaction with Humans
%  Need for concepts/semantic information
%  Dynamic human environment
%  Non-feasibility of hard-coded concepts
%  Requirements to detect novelty and handle it.

% Importance of mapping and localization in a robot as well as reasoning on
% properties of each space.
% Humans give labels to space that characterized the properties and activities
% expected to be performed on them.
% Such information is of great interest to be structured and organized in such
% a way reasoning and planning can be implemented.
%

% Introduction
There has been several efforts in the areas of artificial intelligence and mobile robotics
in creating robots that are able to interact with humans and their environments.
Those robots would be able to move into our houses and offices. They would be able to communicate
with us and provide help solving our tasks.

% Keep listing problems robots should be able to solve 
%\begin{itemize}
%\item Keep listing problems robots should be able to solve
%\item They would dynamically adapt to our environment and exhibit intelligence by understanding
%\item Pose problems robots cannot solve without understanding human environments
%\end{itemize}


% Representation of environments are needed!
In order to reason and understand about the surrounding environment a robot first has to
obtain a manageable representation of it.
Such a representation provides the robot with a understanding on how the several
entities relate to each other, and, if persistent, allow the robot to reason beyond its sensory
horizon.

% Machine-representations are not useful to solve the required problem.
One may argue that robots and computers can already represent and understand environments, but they
do so by using representations that are not human friendly: e.g.\ a robot can uniquely identify the
location of a certain object with coordinates, but those coordinates are not suitable for humans.
A robot may be able to detect a tag placed on an object, but such an approach besides limited
and hard-working does not scales: new objects are created everyday, the huge amount of already
existent objects and concepts also forbids any tagging and concepts by their nature are dynamic.
Summing up: methods are needed that allow robots to represent environments and allow human concepts
to also be representable.

% Semantic Mapping
By gathering semantic knowledge a robot can learn how to map their internal representation to the
given set of human concepts. Web and other common databases, that have been created by humans, are a
great source of semantic information, and, in some sense have helped to solve this semantic mapping
problem.
Common approaches at semantic mapping are based on classification, this is: a robot has a set of
defined categories and it picks the one that best represents the sensed data. This approach often
gives satisfactory results, but only when applied on a controlled environment where the robot has
knowledge of all the categories that exist.

% Complete knowledge? Don't joke me.
This assumption on knowledge of all the categories that an entity can be mapped to is too strong
to hold on unknown environments. In specific, a mapping defined over the human semantics will always
be incomplete due to infeasibility to map all the concepts or even due to its dynamic nature where
concepts change not only on time but also between sociocultural aspects.

% Novelty Detection
In this sense a robot should be able to identify gaps on its knowledge by detecting novelty.
Developing methods for this problem helps to provide robots with novelty signals that can be
used to improve its reliability, decide when to initiate active-learning or even provide basic
information for keeping knowledge continuously adapted in long-term or life-long scenarios.

The rest of this chapter presents in which context is this thesis interested in detecting novelty.
It presents then the specific problem it addresses and defines the goals to achieve.
At last it presents the structure of the other chapters.

\section{Context}
% Spatial Representations
% Spatial Semantics only!
% Room Categories actually.

%
In mobile robotics the main need is on spatial information. Having a representation of all
the spatial entities and relations between them allow the robot to navigate and interact
with them. Maps are the default tool for representing those spaces, and computers excel at
creating close to exact representations of spatial environments.
But which concepts existent in the human spatial knowledge are needed in order to turn those
representations a usable tool to represent human concepts?

Different spatial knowledge types can be identified depending on the scale, abstraction level,
and required generalization. For example geometric type aspects which robots are already very
good at representing, object knowledge which is incredibly basic for humans, delineation
of spatial areas and relations between instances of all those knowledge types. e.g.\ a specific
object (book) is on top of another object (table) that is located inside an area (room-library)
that is inside a another area (university).
As seen on this example humans do not limit themselves at representing entities
in space and spatially relate them but they also attribute semantic value to them.

It is this semantic value attributed by humans to entities that is interesting for endowing
robots with knowledge on how to interpret and reason over human environments.
By understanding which objects are expected to be found on a room, what are the properties that
distinguish room categories, which room types are likely to be connected together a robot can
aim at increase its effectiveness on indoor environments.

A basic concept on indoor environments is that of room. Rooms allow humans to segment areas
in high\hyp{}level entities that can be categorized according to the properties and expected
activities to be performed on them.
e.g.\ kitchens are rooms where cornflakes are usually found, a library is identified by the presence
of many books, a lecture hall is a place where lessons are given.
Since room categories play an important role in understanding and reasoning about
indoor human environments this thesis will focus on them.

\section{Motivation}
% Detect limitations and novel knowledge
It is highly desirable that a robot can be deployed to new and unknown environments. Aiming at that,
a robot has to perform its own semantic mapping of the environment instead of relying on a provision
of already labelled maps.
But the semantic mapping ability of a robot is restricted by its spatial and semantic knowledge.
This poses a problem as its unrealistic to believe that on unknown environments the robot knowledge
is complete. So it becomes important to detect gaps on the robot knowledge, such that the robot can
be aware of limitations of its models and devise techniques to not fail victim of its limitations.

Additionally the detection of knowledge gaps, will serve as base for several other problems in
artificial intelligence such as active-learning and maintenance of long and life-long knowledge.

\section{Proposal and Goals}
This thesis aims at developing methods that can be used to detect gaps on the semantic knowledge
of an agent. For that the spatial knowledge representation proposed in \cite{pronobis2010ias} will
be used as base defining the agent knowledge.
Given an instance world, that knowledge, is used to perform semantic mapping~\cite{pronobis2011semmap}
on indoor environments using a probabilistic graphical model. That model is instantiated from the
environment using the spatial knowledge and captures both semantic and structural information of
space. It allows the agent to map all the modelled variables in the presence of uncertainty and 
therefore this thesis will focus on study and developing methods that can detect novel semantic room
categories from it.

% Study how structure changes novelty.
% If everything goes fine... 2 of the "Future Directions" proposed are 
% Novelty Detection and Learning of Novel Concepts:
%  Identify Gaps in Spatial and Semantic Knowledge -> Addressed
%  Performing learning of new concepts -> Outside Scope / No Time
%
% Using Properties for Space Segmentation
%  If we move on detecting novel structures on the graph we are addressing this
%  issue.


\section{Thesis Outline}
The rest of this thesis is structured as follows:

\autoref{chap:background} introduces background concepts for understanding the
presented thesis. In particular \autoref{sec:graphical-models} introduces
\emph{probabilistic graphical models} that lay the base modelling tool for
the used structured semantic representation.

\autoref{chap:semantic-mapping} describes the semantic mapping process.
It introduces a spatial knowledge representation and describes a system
that uses it for performing semantic mapping using using a \emph{probabilistic graphical model}.

\autoref{chap:novelty-intro} introduces novelty detection under a
statistical view point and how to interpret it as a threshold function.
It presents then the main work of this thesis: detecting novel semantic
categories using graphical models.

\autoref{chap:conclusions} draws conclusions on the developed work and presents
interesting directions for future works.



