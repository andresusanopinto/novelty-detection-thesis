\chapter{Introduction}
% Introduce:
%  Mobile Robotics, A.I.
%  Interaction with Humans
%  Need for concepts/semantic information
%  Dynamic human environment
%  Non-feasibility of hard-coded concepts
%  Requirements to detect novelty and handle it.

% Importance of mapping and localization in a robot as well as reasoning on
% properties of each space.
% Humans give labels to space that characterized the properties and activities
% expected to be performed on them.
% Such information is of great interest to be structured and organized in such
% a way reasoning and planning can be implemented.
%

% Introduction
There has been several efforts in the areas of artificial intelligence and mobile robotics
in creating robots that are able to interact with humans and their environments.
Those robots would be able to explore our houses and offices. They would be able to communicate
with humans and perform typical human-like tasks.

% Keep listing problems robots should be able to solve 
%\begin{itemize}
%\item Keep listing problems robots should be able to solve
%\item They would dynamically adapt to our environment and exhibit intelligence by understanding
%\item Pose problems robots cannot solve without understanding human environments
%\end{itemize}


% Representation of environments are needed!
In order to reason about and perform actions in the surrounding environment, the robot first 
has to obtain a manageable representation of it. Such a representation provides the robot with 
understanding of how the spatial entities relate to each other and spatial concepts, and, 
if persistent, allow the robot to reason beyond its sensory horizon.

% Machine-representations are not useful to solve the required problem.
One may argue that robots and computers can already represent and understand environments, but they
do so by using representations that are not human friendly: e.g.\ a robot can uniquely identify the
location of a certain landmark using internal coordinates, but those coordinates are not understandable
for humans. A robot may be able to detect a tag placed on an object, but such an approach does not scale
outsize structured environments: new objects are created everyday, there is huge amount of already
existing objects and concepts by their nature are dynamic. For this reason, methods need 
to be developed that will allow robots to represent real-world environments and allow human concepts
to be transferred into the robot representations.

% Semantic Mapping
By gathering semantic knowledge a robot can learn how to augment the internal representation with a
set of human concepts. Web and other common databases, that have been created by humans, are a
great source of semantic information, and can be a valuable source of knowledge for the semantic mapping
problem. Common approaches to semantic mapping are based on classification, i.e., a robot has a set of
defined categories and selects the one that best represents the sensed data. This approach often
gives satisfactory results, but only when applied in a controlled environment where the robot has
knowledge about all of the categories that exist.

% Complete knowledge? Don't joke me.
This assumption about the complete knowledge of all the categories that an entity can be mapped to is 
too strong to hold on real-world environments. More specifically, a mapping defined over the human semantics 
will always be incomplete due to the infeasibility to map all the concepts or even due to its dynamic nature 
where concepts change not only over time but also between sociocultural aspects.

% Novelty Detection
Consequently, a robot should be able to identify gaps in its own knowledge and detect novelty.
Developing methods for this problem helps to provide robots with novelty signals that can be
used to improve reliability, decide when to initiate active learning or even provide basic
information for keeping knowledge continuously adapted in long-term or life-long scenarios.

The rest of this chapter presents the context in which this thesis addresses the problem of novelty detection.
Then, it presents the specifics of the problem and defines the goals to achieve. Finally, it outlines the 
structure of the remaining chapters.


\section{Context}
% Spatial Representations
% Spatial Semantics only!
% Room Categories actually.

%
One of the most crucial problems in mobile robotics is that of representing space in which the robot operates. 
Having a representation of all the spatial entities and relations between them allows the robot to navigate and 
interact with the environment. Maps are the default tool for representing those spaces, and computers excel at
creating close to exact representations of spatial environments. However, those representations usually fail to
capture aspects important for human-like spatial understanding.

Different spatial knowledge types can be identified depending on the scale, abstraction level,
and required generalization. Examples include geometric aspects of the world, object knowledge 
(which is fundamental for humans, but difficult to extract for robots), segmentation of
spatial areas, and finally, relations between different spatial entities e.g. a specific
object (book) is on top of another object (table) that is located inside an area (room-library)
that is inside a another area (university). We see that humans do not limit themselves to representing 
entities of space and relating them spatially, but they also attribute semantic concept to them.

It is those semantic concepts attributed by humans to spatial entities that are interesting for 
endowing robots with knowledge on how to interpret and reason about human environments.
By understanding which objects are expected to be found in a room, what are the properties that
distinguish room categories, or which room types are likely to be connected together, a robot can
increase its communication abilities but also its performance on complex tasks in indoor environments.

A basic concept of indoor environments is that of a room. Rooms allow humans to segment areas
into high\hyp{}level entities that can be categorized according to their properties and canonical
activities performed there, e.g. kitchens are rooms where cornflakes are usually found, a library is 
identified by the presence of many books, a lecture hall is a place where lessons are given.
Since room categories play an important role in understanding and reasoning about
indoor human environments this thesis will focus on them.


\section{Motivation}
% Detect limitations and novel knowledge
It is highly desirable that a robot can be deployed in new and unknown environments. In order to achieve that,
the robot has to perform its own semantic mapping of the environment instead of relying on available labelled maps.
However, the semantic mapping ability of a robot is restricted by its spatial and semantic knowledge.
This poses a problem as its unrealistic to assume that in an unknown environment, the robot's knowledge
will be complete. Therefore, it becomes important to detect gaps in the robot's knowledge so that the robot can
be aware of the limitations of its models and deal with the novel situations. Furthermore, the detection of 
knowledge gaps can be extremely useful for several other problems in artificial intelligence such as 
active learning and knowledge maintenance during life-long operation.

\section{Goals}
This thesis aims at developing methods that can be used to detect gaps in spatial semantic knowledge
of a robotic agent. To this end, the spatial knowledge representation proposed in \cite{pronobis2010ias} 
is used as a base for defining the agent's spatial knowledge. Given a concrete environment, 
a semantic mapping process is applied and a conceptual map representation of the environment it build according
to~\cite{pronobis2011semmap}. The conceptual map is represented in terms of a probabilistic graphical model
capturing both semantic and structural information of space. This thesis focuses on novelty detection methods 
from this probabilistic graphical model representation. In particular, emphasis is placed on detection of 
novel room categories given the observed spatial properties of the environment.

% Study how structure changes novelty.
% If everything goes fine... 2 of the "Future Directions" proposed are 
% Novelty Detection and Learning of Novel Concepts:
%  Identify Gaps in Spatial and Semantic Knowledge -> Addressed
%  Performing learning of new concepts -> Outside Scope / No Time
%
% Using Properties for Space Segmentation
%  If we move on detecting novel structures on the graph we are addressing this
%  issue.


\section{Thesis Outline}
The rest of this thesis is structured as follows:

\autoref{chap:background} introduces fundamental concepts important for understanding the problems 
addressed in the thesis. In particular \autoref{sec:graphical-models} introduces \emph{probabilistic 
graphical models} that lay the foundation for the conceptual map.

\autoref{chap:semantic-mapping} describes the semantic mapping process. It introduces the spatial 
knowledge representation and describes a system that uses it for performing semantic mapping using 
using the \emph{probabilistic graphical model}.

\autoref{chap:novelty-intro} introduces novelty detection from the statistical point of view and 
shows how to interpret it as a thresholding function. Then, it the main contribution of this thesis:
methods for detecting novel semantic categories using graphical models.

\autoref{chap:conclusions} draws conclusions on the developed work and presents possible 
directions for future works.



