\chapter{Introduction} \label{chap:intro}

\section*{}

AO primeiro capítulo da dissertação deve servir para apresentar o
enquadramento e a moti\-va\-ção do trabalho e para identificar e
definir os problemas que a dissertação aborda.
Deve resumir as metodologias utilizadas no trabalho e termina
apresentando um breve resumo de cada um dos capítulos
posteriores.

Este documento ilustra o formato a usar em dissertações na \gls{FEUP}, não
servindo de exemplo sobre os conteúdos a usar.
São dados exemplos de margens, cabeçalhos, títulos, paginação, estilos
de índices, etc. 
São ainda dados exemplos de formatação de citações, figuras e tabelas,
equações, referências cruzadas, lista de referências e índices.

Uma recolha de normas existentes sobre este assunto pode ser
encontrada em~\cite{kn:Mat93}. 

\begin{quote}
  ``Like the Abstract, the Introduction should be written to engage the
  interest of the reader. It should also give the reader an idea of
  how the dissertation is structured, and in doing so, define the
  thread of the contents.''~\cite{kn:Tha01} 
\end{quote}

Blabla \gls{Kinect}

\section{Context} \label{sec:context}

Esta secção descreve a área em que o trabalho se insere, podendo
referir um eventual projecto de que faz parte e apresentar uma breve
descrição da empresa onde o trabalho decorreu.


\section{Motivation and Goals} \label{sec:goals}

Apresenta a motivação e enumera os objectivos do trabalho terminando
com um resumo das metodologias para a prossecução dos objectivos.


\section{Thesis Outline} \label{sec:outline}

The rest of this thesis is structured as follows\dots

