\chapter{Introduction}
% Introduce:
%  Mobile Robotics, A.I.
%  Interaction with Humans
%  Need for concepts/semantic information
%  Dynamic human environment
%  Non-feasibility of hard-coded concepts
%  Requirements to detect novelty and handle it.

% Importance of mapping and localization in a robot as well as reasoning on
% properties of each space.
% Humans give labels to space that characterized the properties and activities
% expected to be performed on them.
% Such information is of great interest to be structured and organized in such
% a way reasoning and planning can be implemented.
%

% Introduction
There has been several efforts in the areas of artificial intelligence and mobile robotics
in creating robots that are able to interact with humans and their environments.
Those robots would be able to move into our houses and offices. They would be able to communicate
with us and provide help solving our tasks.

% Keep listing problems robots should be able to solve 
\begin{itemize}
\item Keep listing problems robots should be able to solve
\item They would dynamically adapt to our environment and exhibit intelligence by understanding
\item Pose problems robots cannot solve without understanding human environments
\end{itemize}


% Representation of environments are needed!
In order to reason and understand about the surrounding environment a robot first has to
obtain a manageable representation of it.
Such a representation provides the robot with a understanding on how the several
entities relate to each other, and, if persistent, allow the robot to reason beyond its sensory
horizon.

% Machine-representations are not useful to solve the required problem.
One may argue that robots and computers can already represent and understand environments, but they
do so by using representations that are not human friendly: e.g.\ a robot can uniquely identify the
location of a certain object with coordinates, but those coordinates are not suitable for humans.
A robot may be able to detect a tag placed on an object, but such an approach besides limited
and hard-working does not scales: new objects are created everyday, the huge amount of already
existent objects and concepts also forbids any tagging and concepts by their nature are dynamic.
Summing up: methods are needed that allow robots to represent environments and allow human concepts
to also be representable.

% Semantic Mapping
By gathering semantic knowledge a robot can learn how to map their internal representation to the
given set of human concepts. Web and other common databases, that have been created by humans, are a
great source of semantic information, and, in some sense have helped to solve this semantic mapping
problem.
Common approaches at semantic mapping are based on classification, this is: a robot has a set of
defined categories and it picks the one that best represents the sensed data. This approach often
gives satisfactory results, but only when applied on a controlled environment where the robot has
knowledge of all the categories that exist.

% Complete knowledge? Don't joke me.
This assumption on knowledge of all the categories that an entity can be mapped to is too strong
to hold on unknown environments. In specific, a mapping defined over the human semantics will always
be incomplete due to infeasibility to map all the concepts or even due to its dynamic nature where
concepts change not only on time but also between sociocultural aspects.

% Novelty Detection
In this sense a robot should be able to identify gaps on its knowledge by detecting novelty.
Developing methods for this problem helps to provide robots with novelty signals that can be
used to improve its reliability, decide when to initiate active-learning or even provide basic
information for keeping knowledge continuously adapted in long-term or life-long scenarios.

The rest of this chapter presents in which context is this thesis interested in detecting novelty.
It presents then the specific problem it addresses and defines the goals to achieve.
At last it presents the structure of the other chapters.

\section{Context}
% Spatial Semantics only!
% Mobile Robotics
% Room Categories

Being able to deal with any 
in this sense this thesis will focus only on the semantic spatial mapping problem.

Humans organize spaces as areas, they characterize them by the type of 

% Yeah thats what you call dream :P
In an extreme ideal case the agent should be able to go from zero-knowledge to
understanding presence of certain objects (as generators of a set of sensed
sensed properties grouped locally), understand areas within rooms (sink-area on
on a kitchen), rooms as separated by doors, and understand environments such
as office, home, warehouse, spaceship.

% Semantic knowledge (spatial-knowledge)
% It also holds an important role in long-term planning and stability as it
% hides the space-time local details and allows to focus on high-level thinking.

In that sense the need for semantic mapping arises, and it can be described as the
relation between spatial areas and a set of existent concepts used by humans to
describe them.

, not just for the purpose of navigation and obstacle avoidance, 
but also in terms of human semantics and functionality.

An essential competence for cognitives robots is their ability to reason about space.
Spatial knowledge plays an important role in navigation, reasoning, planning and episodic memory.
% == Introduce the need of semantic representations ==
%
% There is a lot of low-level methods (or specific knowledge to robots)
% But the creation of semantic representations allow to bridge that low-level
% sensing with high-level concepts that facilitate several high-level Tasks.




% Spatial knowledge representation
% Semantic Mapping



\section{Motivation}
% Detect limitations and novel knowledge

Deploy to new environments, long-term and active learning,
understand of limitations of the existent knowledge
reliability

Example of what kind of an high-level thinking on space and its semantics can
provide: Get me a beer/milk concept.


% == Reasoning about need for Novelty Detection ==
%
% Dynamic environment (inability to know the environment or even to be possible
% to map all semantic concepts to interact with humans).
%
% Knowledge-Awareness (knowing what we know... And detecting novel cases)
% Identify gaps in the semantic knowledge.
% Automatic detection and learning of novel concepts.
%
% Increase robustness
% Self-Extending
%

\section{Problem}
% Deal with uncertainty -> need for probabilistic models.
% Deal with semantic    -> need of structured models such as graphical models.


They need to possess ability to adapt to situations as its infeasible to rely
on extensive man-work to tag objects, map space and code all the aspects that
make up our human-reality.


Lack of methods that are able to detect gaps on their
Study how structure changes novelty

\section{Proposal}

% Extend a probabilistic graphical modelling framework with capabilities to
% detect novelty.
% Both in terms of novel classes as of novel structure.
%
Given a probabilistic structure representing the sensed conceptual knowledge
obtained by the agent develop methods able to detect novelty present
either on new semantic concepts (new classes) or even on structure that was
previously unknown.


For that the probabilistic semantic representation presented by
\cite{pronobis2011phd} is used.

% If everything goes fine... 2 of the "Future Directions" proposed are 
% Novelty Detection and Learning of Novel Concepts:
%  Identify Gaps in Spatial and Semantic Knowledge -> Addressed
%  Performing learning of new concepts -> Outside Scope / No Time
%
% Using Properties for Space Segmentation
%  If we move on detecting novel structures on the graph we are addressing this
%  issue.



\section{Thesis Outline}
The rest of this thesis is structured as follows:

\autoref{chap:background} introduces background concepts for understanding the
presented thesis. In particular \autoref{sec:graphical-models} introduces
\emph{probabilistic graphical models} that lay the base modelling tool for
the used structured semantic representation.

\autoref{chap:semantic-mapping} describes the semantic mapping process.
It introduces a spatial knowledge representation and describes a system
that uses it for performing semantic mapping using using a \emph{probabilistic graphical model}.

\autoref{chap:novelty-intro} introduces novelty detection under a
statistical view point and how to interpret it as a threshold function.
It presents then the main work of this thesis: detecting novel semantic
categories using graphical models.

\autoref{chap:conclusions} draws conclusions on the developed work and presents
interesting directions for future works.



